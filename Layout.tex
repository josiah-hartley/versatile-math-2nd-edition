\usepackage[svgnames,table]{xcolor}
\usepackage{circuitikz}
\usepackage{xparse}
\usepackage{filecontents}
%\usepackage{minibox}
\usepackage{longtable}
\usepackage{amsmath,pgfplots}
\usepgfplotslibrary{statistics}
\usepackage[shortlabels]{enumitem}
%\usepackage{tkz-euclide}
%\usetkzobj{all}

\usepackage{tikz}
\usepackage{tkz-berge}

\usetikzlibrary{decorations.fractals,calc,decorations.markings}
\usetikzlibrary{arrows,decorations.markings,positioning}
\newcommand{\tikzmark}[2]{\tikz[overlay, remember picture] \node[inner sep=5pt, outer sep=5pt, anchor=base] (#1) {#2};}

\usepackage{amssymb}
\usepackage{etoolbox}
\usepackage{cancel}
\usepackage{array}
\usepackage{graphicx}
\usepackage{framed}
\usepackage{tcolorbox}
\tcbuselibrary{breakable,skins,xparse}
\usepackage{marginnote}
%\usepackage{verbatim}
\usepackage[style=2]{mdframed}
\usepackage[top=0.75in, bottom=0.75in, inner=0.75in, outer=2.5in, marginparwidth=1.75in, marginparsep=0.25in]{geometry}
\usepackage{titlesec}
\usepackage[official]{eurosym}
%\usepackage[Bjornstrup]{fncychap}

\usepackage{changepage}
\usepackage[T1]{fontenc}
\usepackage{lmodern}
%\usepackage{url}
\usepackage{comment}
\ifpdf
  \usepackage{pdfcolmk}
\fi
\usepackage{graphicx}
\usepackage[hidelinks]{hyperref}
%\usepackage[outline]{contour}
\usepackage{pifont}
%\ifxetex
%  \usepackage{fontspec}
%\fi

\strictpagecheck
\addtocontents{toc}{\protect\thispagestyle{empty}}

%%%%Author Name in Table of Contents---------
\makeatletter
\let\@nodottedtocline\@dottedtocline
\patchcmd{\@nodottedtocline}{\hbox{.}}{\hbox{}}{}{}
\patchcmd{\@nodottedtocline}{\normalcolor #5}{\normalcolor}{}{}
\newcommand*\l@sectionsubtitle{\@nodottedtocline{1}{0em}{1.5em}}
\makeatother

\def\sectionsubtitle#1{%
    \addcontentsline{toc}{sectionsubtitle}{\protect\numberline{}\textit{#1}}%
}
%%%%------------------------------------------

%%%%Define Gaussian Plot for Stats Chapter
\pgfmathdeclarefunction{gauss}{2}{%
  \pgfmathparse{1/(#2*sqrt(2*pi))*exp(-((x-#1)^2)/(2*#2^2))}%
}
%%%%End Define Gaussian Plot

%%%%Drawing Fractal
\newcommand{\bedknobsthree}[1]{
    %\begin{center}
        \tikz [
            decoration = #1
        ] \draw decorate{ decorate{ decorate{ (0, 0) -- (0.35\linewidth, 0) }}};
    %\end{center}
}
%%%%Drawing Fractal

%%%%Begin Header Format
\nouppercaseheads
\copypagestyle{doc}{companion}
\makeevenhead{doc}{\textbf{\Large \thepage} \hspace{0.1in} \textbf{CHAPTER \thechapter} \hspace{0.1in} \leftmark}{}{}
\makeoddhead{doc}{}{}{SECTION \rightmark \hspace{0.1in} \textbf{\Large \thepage}}
\makeevenfoot{doc}{}{}{}
\makeoddfoot{doc}{}{}{}
\makeheadrule{doc}{\textwidth}{0pt} %Uncomment to remove headrule

\copypagestyle{algebra}{doc}
\makeevenhead{algebra}{\textbf{\Large \thepage} \hspace{0.1in} \textbf{CHAPTER R \hspace{0.1in} \leftmark}}{}{}

\copypagestyle{chapter}{doc}
\makeevenhead{chapter}{}{}{}
\makeoddhead{chapter}{}{}{}
\makeevenfoot{chapter}{\textbf{\Large \thepage}}{}{}
\makeoddfoot{chapter}{\hfill}{\hfill}{\textbf{\Large \thepage}}
\makeheadrule{chapter}{\textwidth}{0pt} %Uncomment to remove headrule

\copypagestyle{exercise}{doc}
\makeevenfoot{exercise}{
\begin{tikzpicture}[overlay,remember picture]
      \fill [color=blue!60!black]
        ($ (current page.north west) - (1cm,-1cm) $)
        rectangle
        ($ (current page.south west) + (1cm,-1cm) $);
\end{tikzpicture}
}{}{}
\makeoddfoot{exercise}{\hfill}{\hfill}{
\begin{tikzpicture}[overlay,remember picture]
      \fill [color=blue!60!black]
        ($ (current page.north east) + (1cm,1cm) $)
        rectangle
        ($ (current page.south east) - (1cm,1cm) $);
\end{tikzpicture}
}

\setsecheadstyle{\Large\scshape\raggedright\bfseries\color{blue!30!black}}
\setsubsecheadstyle{\large\scshape\raggedright}
\setsubsubsecheadstyle{\normalsize\scshape\raggedright}
\setbeforesecskip{-1.5ex plus -.5ex minus -.2ex}
\setaftersecskip{1.3ex plus .2ex}
\setbeforesubsecskip{-1.25ex plus -.5ex minus -.2ex}
\setaftersubsecskip{1ex plus .2ex}
%%%%End Header Format

%%%%Begin Chapter Header Format
\usepackage{color,calc}
\newsavebox{\ChpNumBox}
\definecolor{ChapBlue}{rgb}{0.2,0.6,1}
\makeatletter
%\newcommand*{\thickhrulefill}{%
%\leavevmode\leaders\hrule height 1\p@ \hfill \kern \z@}
\newcommand*{\thickhrulefill}{\rule{1.0\textwidth}{5pt}}
\newcommand*\BuildChpNum[2]{%
\begin{tabular}[t]{@{}c@{}}
\makebox[0pt][c]{#1\strut} \\[.5ex]
\colorbox{ChapBlue}{%
\rule[-10em]{0pt}{0pt}%
\rule{1ex}{0pt}\color{black}#2\strut
\rule{1ex}{0pt}}%
\end{tabular}}
\makechapterstyle{BlueBox}{%
\renewcommand{\chapnamefont}{\large\scshape}
\renewcommand{\chapnumfont}{\Huge\bfseries}
\renewcommand{\chaptitlefont}{\raggedright\Huge\bfseries}
\setlength{\beforechapskip}{20pt}
\setlength{\midchapskip}{26pt}
\setlength{\afterchapskip}{40pt}
\renewcommand{\printchaptername}{}
\renewcommand{\chapternamenum}{}
\renewcommand{\printchapternum}{%
\sbox{\ChpNumBox}{%
\BuildChpNum{\chapnamefont\@chapapp}%
{\chapnumfont\thechapter}}}
\renewcommand{\printchapternonum}{%
\sbox{\ChpNumBox}{%
\BuildChpNum{\chapnamefont\vphantom{\@chapapp}}%
{\chapnumfont\hphantom{\thechapter}}}}
\renewcommand{\afterchapternum}{}
\renewcommand{\printchaptertitle}[1]{%
\usebox{\ChpNumBox}\hfill
\parbox[t]{\hsize-\wd\ChpNumBox-1em}{%
\vspace{\midchapskip}%
\thickhrulefill\par
\chaptitlefont ##1\par}}%
}
\chapterstyle{BlueBox}
%%%%End Chapter Header Format

%%%%Begin Title Page Definition
\newlength{\drop}
\newcommand*{\FSfont}[1]{%
  \fontencoding{T1}\fontfamily{#1}\selectfont}
\renewcommand*{\FSfont}[1]{}%    kills special font selections

\newcommand*{\rotrt}[1]{\rotatebox{90}{#1}}
\newcommand*{\rotlft}[1]{\rotatebox{-90}{#1}}
\newcommand*{\topb}{%
  \resizebox*{\unitlength}{\baselineskip}{\rotrt{$\}$}}}
\newcommand*{\botb}{%
  \resizebox*{\unitlength}{\baselineskip}{\rotlft{$\}$}}}
\newcommand*{\titleBC}{\begingroup
\begin{center}
\settowidth{\unitlength}{\textit{\Large COMMON APPLICATIONS OF MATHEMATICS}} 
{\color{LightGoldenrod}\topb} \\[\baselineskip]
\textcolor{Sienna}{\textit{\HUGE Versatile Mathematics}} \\[\baselineskip]
{\color{RosyBrown}\Large COMMON MATHEMATICAL APPLICATIONS} \\
{\color{LightGoldenrod}\botb}
\end{center}
\vfill
\begin{center}
{\LARGE\textbf{Josiah Hartley}}\\ Frederick Community College\\ \text{}\\ \text{}\\
{\LARGE\textbf{Val Lochman}}\\ Frederick Community College\\ \text{}\\ \text{}\\
{\LARGE\textbf{Erum Marfani}}\\ Frederick Community College\\ \text{}\\ \text{}\\
\vfill
{\huge\textbf{2nd Edition}}\\ \text{}\\ \text{}\\
{\LARGE\textbf{2020}}
\end{center}
\endgroup}

\newcommand*{\titleSolutionManual}{\begingroup
\begin{center}
\settowidth{\unitlength}{\textit{\Large COMMON APPLICATIONS OF MATHEMATICS}} 
{\color{LightGoldenrod}\topb} \\[\baselineskip]
\textcolor{Sienna}{\textit{\HUGE Versatile Mathematics}} \\[\baselineskip]
{\color{RosyBrown}\Large COMMON MATHEMATICAL APPLICATIONS} \\
{\color{LightGoldenrod}\botb}
\end{center}
\vfill
\begin{center}
{\HUGE\textbf{Instructor Solution Manual}}\\ \text{}\\
{\LARGE\textbf{NOT FOR USE BY STUDENTS}}
\end{center}
\vfill
\begin{center}
{\LARGE\textbf{Josiah Hartley}}\\ Frederick Community College\\ \text{}\\ \text{}\\
{\LARGE\textbf{Val Lochman}}\\ Frederick Community College\\ \text{}\\ \text{}\\
{\LARGE\textbf{Erum Marfani}}\\ Frederick Community College\\ \text{}\\ \text{}\\
\vfill
{\huge\textbf{2nd Edition}}\\ \text{}\\ \text{}\\
{\LARGE\textbf{2020}}
\end{center}
\endgroup}
%%%%End Title Page Definition

%Makes labels on axes smaller
\pgfplotsset{
    every axis/.append style={font=\tiny},  
}
%

%%%%Begin Procedure Definition
\newtcolorbox{proc}[1]{enhanced,colback=green!5!white,colframe=green!40!white!40!black,drop fuzzy shadow, toprule=5pt,bottomrule=0mm,rightrule=0mm,leftrule=0mm,fonttitle=\large,title={\fontfamily{fvs}\selectfont \textbf{#1}},sharp corners=all}
%%%%End Procedure Definition

\def\fra#1#2{\vskip #2\noindent\textcolor{black!60!red}{\rule{\textwidth}{2pt}}}

%%%%Begin Try It Definition
\renewcommand{\pfbreakdisplay}{%
\ding{167}\quad\ding{167}\quad\ding{167}}

\newenvironment{try}[1][]
{
\begin{tcolorbox}[enhanced,beforeafter skip=10pt,breakable,colframe=yellow!10,colback=yellow!10,drop lifted shadow,sharp corners=all,hyperurl=#1]
\marginnote{\href{#1}{\fontfamily{fvs}\selectfont \Large\color{black!70!orange}\textbf{TRY IT}}}
}
{ \end{tcolorbox}}

\newenvironment{trynolabel}[1][]
{
\begin{tcolorbox}[enhanced,beforeafter skip=10pt,breakable,colframe=yellow!10,colback=yellow!10,drop lifted shadow,sharp corners=all,hyperurl=#1]
}
{ \end{tcolorbox}}
%%%%End Try It Definition

%%%%Begin Formula Definition
\newenvironment{formula}[1]
{\begin{tcolorbox}[beforeafter skip=10pt,colframe=black!10,colback=blue!70!cyan!20!white,opacityfill=0.5,toprule=0mm,bottomrule=0mm,rightrule=0mm,leftrule=0mm,sharp corners=all]

{\fontfamily{fvs}\selectfont \Large\bfseries #1}\\

}
{\end{tcolorbox}}
%%%%End Formula Definition

%%%%Counter for examples; reset for each chapter/section?
\newcounter{ExampleCounter}
\setcounter{ExampleCounter}{1}

%%%%Begin Example Definition
\begin{comment}
\newenvironment{example}[2][https://dl.dropboxusercontent.com/u/28928849/Webpages/MathInSocietyVideos2.htm]
{
\marginnote{\href{#1}{\fontfamily{fvs}\selectfont \Large\color{black!70!blue!80!cyan}\uppercase{\bfseries\centering EXAMPLE \theExampleCounter}
}}[9.2mm]
\begin{tcolorbox}[check odd page,toggle left and right,
beforeafter skip=10pt,breakable,bottomrule at break=0mm,toprule at break=0mm,colframe=black!50!blue!50!cyan,colback=white,toprule=0mm,rightrule=0mm,sharp corners=all] 
{%\renewcommand*\sfdefault{iwona}
\checkoddpage
\ifoddpage{
\hfill {\href{#1}{\fontfamily{fvs}\selectfont \Large\color{black!70!blue!80!cyan}\uppercase{\bfseries #2}}}}\\
 \else{
{\href{#1}{\fontfamily{fvs}\selectfont \Large\color{black!70!blue!80!cyan}\uppercase{\bfseries #2}}}}\\
\fi
\end{comment}

\colorlet{colexam}{black!50!blue!50!cyan}
\newenvironment{example}[2][]
{
\begin{tcolorbox}[empty,
breakable,
if odd page*={overlay unbroken={\coordinate (X) at ([yshift=-10pt]frame.north west); \coordinate (Y) at (frame.south west); \coordinate (Z) at (frame.south east);
\draw[black!50!blue!50!cyan,line width=2pt]
(X)--(Y)--(Z); },
overlay first={\coordinate (X) at ([yshift=-10pt]frame.north west); \coordinate (Y) at (frame.south west);
\draw[black!50!blue!50!cyan,line width=2pt]
(X)--(Y); },
overlay middle={\coordinate (X) at ([yshift=-10pt]frame.north west); \coordinate (Y) at (frame.south west);
\draw[black!50!blue!50!cyan,line width=2pt]
(X)--(Y); },
overlay last={\coordinate (X) at ([yshift=-10pt]frame.north west); \coordinate (Y) at (frame.south west); \coordinate (Z) at (frame.south east);
\draw[black!50!blue!50!cyan,line width=2pt]
(X)--(Y)--(Z); }}{overlay unbroken={\coordinate (X) at ([yshift=-10pt]frame.north east); \coordinate (Y) at (frame.south east); \coordinate (Z) at (frame.south west);
\draw[black!50!blue!50!cyan,line width=2pt]
(X)--(Y)--(Z); },
overlay first={\coordinate (X) at ([yshift=-10pt]frame.north east); \coordinate (Y) at (frame.south east);
\draw[black!50!blue!50!cyan,line width=2pt]
(X)--(Y); },
overlay middle={\coordinate (X) at ([yshift=-10pt]frame.north east); \coordinate (Y) at (frame.south east);
\draw[black!50!blue!50!cyan,line width=2pt]
(X)--(Y); },
overlay last={\coordinate (X) at ([yshift=-10pt]frame.north east); \coordinate (Y) at (frame.south east); \coordinate (Z) at (frame.south west);
\draw[black!50!blue!50!cyan,line width=2pt]
(X)--(Y)--(Z); }}] 
{%\renewcommand*\sfdefault{iwona}
\checkoddpage
\ifoddpage{
\hfill {\href{#1}{\fontfamily{fvs}\selectfont \Large\color{black!70!blue!80!cyan}\uppercase{\bfseries #2}\marginnote{\href{#1}{\fontfamily{fvs}\selectfont \Large\color{black!70!blue!80!cyan}\uppercase{\bfseries\centering EXAMPLE \theExampleCounter}
}}}}}\\
 \else{
{\href{#1}{\fontfamily{fvs}\selectfont \Large\color{black!70!blue!80!cyan}\uppercase{\bfseries #2}\marginnote{\href{#1}{\fontfamily{fvs}\selectfont \Large\color{black!70!blue!80!cyan}\uppercase{\bfseries\centering EXAMPLE \theExampleCounter}
}}}}}\\
\fi
%\renewcommand*\sfdefault{cmr}
}

%\Large\color{black!70!blue!80!cyan}\scshape #2}\\

}
{ \end{tcolorbox} \stepcounter{ExampleCounter}}

\newcommand{\sol}{
\begin{center}\textcolor{black!50!blue!50!cyan}{\line(1,0){250}}\end{center}
\marginnote{\bfseries Solution}
}
\newcommand{\solline}{
\begin{center}\textcolor{black!50!blue!50!cyan}{\line(1,0){250}}\end{center}
}
%%%%End Example Definition

\newcommand\calcbutton[1]{$\boxed{\textrm{#1}}$}
\newcommand\boxtext[1]{$\boxed{\textrm{#1}}$}

%%%%Begin Section Definition
\newcounter{SectionNo}[chapter]

\DeclareDocumentCommand \section { m }
{
\phantomsection
\stepcounter{section}
\addcontentsline{toc}{section}{\thesection\ #1}
\sectionmark{#1}

\checkoddpage
\ifoddpage{\begin{adjustwidth}{0in}{-1.75in}
\stepcounter{SectionNo}
	
	\tcbset{enhanced,colback=brown!70!white,colframe=brown!75!black}
	\begin{tcolorbox}[toprule=3mm,sharp corners = all]
	{\fontfamily{fvs}\selectfont \huge\bfseries\color{white} SECTION \thechapter .\theSectionNo} \ {\fontfamily{fvs}\selectfont \LARGE\bfseries\color{black!80} #1}
	\end{tcolorbox}
	
 \end{adjustwidth}}
\else{\begin{adjustwidth}{-1.75in}{0in}
\stepcounter{SectionNo}
	
	\tcbset{enhanced,colback=brown!70!white,colframe=brown!75!black}
	\begin{tcolorbox}[toprule=3mm,sharp corners = all]
	{\fontfamily{fvs}\selectfont \huge\bfseries\color{white} SECTION \thechapter .\theSectionNo} \ {\fontfamily{fvs}\selectfont \LARGE\bfseries\color{black!80} #1}
	\end{tcolorbox}
	
 \end{adjustwidth}}
\fi
}

\DeclareDocumentCommand \subsection { m }
{
\vspace{0.2in}
{\fontfamily{fvs}\selectfont \LARGE\color{brown!65!black} \textbf{#1}}
\vspace{0.1in}\\
}
%%%%End Section Definition

%%%%Begin Example 2.0 Definition
\newcounter{example}[SectionNo]

\DeclareDocumentCommand \examplet { O{www.google.com} +m O{www.frederick.edu} +m }
{
\stepcounter{example}
\checkoddpage
\ifoddpage{\begin{adjustwidth}{0in}{-1.75in}
\begin{tcolorbox}[enhanced,breakable,drop lifted shadow,sharp corners=all,sidebyside,sidebyside align=top seam,title=\href{#1}{{\Large\bfseries\color{blue!40!black}EXAMPLE \theexample}} \hfill \href{#3}{{\Large\bfseries\color{green!40!black} TRY IT}},colframe=white,bottomrule=0mm,righthand width=2in,bicolor,colback=brown!15!white,colbacklower=yellow!10] #2 \tcblower #4
\end{tcolorbox}
\end{adjustwidth}}
\else{
\begin{adjustwidth}{-1.75in}{0in}
\begin{tcolorbox}[enhanced,breakable,drop lifted shadow,sharp corners=all,sidebyside,sidebyside align=top seam,title=\href{#1}{{\Large\bfseries\color{blue!40!black}EXAMPLE \theexample}} \hfill \href{#3}{{\Large\bfseries\color{green!40!black} TRY IT}},colframe=white,bottomrule=0mm,righthand width=2in,bicolor,colback=brown!15!white,colbacklower=yellow!10] #2 \tcblower #4
\end{tcolorbox}
\end{adjustwidth}}
\fi
}

\begin{comment}
\DeclareDocumentCommand \example { O{www.google.com} +m }
{
\stepcounter{example}
\checkoddpage
\ifoddpage{\begin{adjustwidth}{0in}{-1.75in}
\begin{tcolorbox}[enhanced,breakable,drop lifted shadow,sharp corners=all,title=\href{#1}{{\Large\bfseries\color{blue!40!black}EXAMPLE \theexample}},colframe=white,bottomrule=0mm,colback=brown!15!white] #2
\end{tcolorbox}
\end{adjustwidth}}
\else{
\begin{adjustwidth}{-1.75in}{0in}
\begin{tcolorbox}[enhanced,breakable,drop lifted shadow,sharp corners=all,title=\href{#1}{{\Large\bfseries\color{blue!40!black}EXAMPLE \theexample}},colframe=white,bottomrule=0mm,colback=brown!15!white] #2
\end{tcolorbox}
\end{adjustwidth}}
\fi
}
\end{comment}
%%%%End Example 2.0 Definition

%%%%Begin Objectives Definition
%\newenvironment{objectives}
%{\begin{adjustwidth}{-1.5in}{0in}\begin{minipage}{0.5\textwidth}}
%{\end{minipage}\end{adjustwidth}}
\newenvironment{objectives}
{\marginnote{\begin{enumerate}}
{\end{enumerate}}}
%%%%End Objectives Definition

%%%%Begin Exercises Definition
\newcounter{ProblemNumber}

\newcommand\pone[1]{ %One exercise per line
  \noindent
  \begin{minipage}[t]{\textwidth}
  \vspace{0.1in}
    {\large\bfseries\theProblemNumber .} #1
  \vspace{0.1in}
  \end{minipage}
  \stepcounter{ProblemNumber}
}

\newcommand\ptwo[1]{ %Two exercises per line
  \noindent
  \begin{minipage}[t]{0.48\textwidth}
  \vspace{0.1in}
    {\large\bfseries\theProblemNumber .} #1
  \vspace{0.1in}
  \end{minipage}
  \stepcounter{ProblemNumber}
}

\newcommand\pthree[1]{ %Three exercises per line
  \noindent
  \begin{minipage}[t]{0.31\textwidth}
  \vspace{0.1in}
    {\large\bfseries\theProblemNumber .} #1
  \vspace{0.1in}
  \end{minipage}
  \stepcounter{ProblemNumber}
}

\newcommand\pfour[1]{ %Four exercises per line
  \noindent
  \begin{minipage}[t]{0.228\textwidth}
  \vspace{0.15in}
    {\large\bfseries\theProblemNumber .} #1
  \vspace{0.15in}
  \end{minipage}
  \stepcounter{ProblemNumber}
}

\newenvironment{exercises}
{ 
	\newgeometry{margin=0.75in}	
	\pagestyle{exercise}
	\setcounter{ProblemNumber}{1}
	\noindent{\href{https://www.myopenmath.com/index.php}{\Huge\bfseries\color{blue!60!black} Exercises \thesection}}\\
	\checkoddpage
	\ifoddpage{{\color{blue!60!black} \rule{1.5\textwidth}{3pt}}} 
	\else{\begin{adjustwidth}{-0.8in}{0in}{\color{blue!60!black} \par\noindent\hspace{-0.25\textwidth}\rule{1.75\textwidth}{3pt}}\end{adjustwidth}} 
	\fi
	%{\color{blue!60!black} \rule{1.1\textwidth}{3pt}}
	

%\begin{comment}
%\tikz[remember picture,overlay] {%
%\draw [blue!60!black,line width=10mm]
%(current page.south east)
%rectangle
%(current page.north east)
%;}
%\end{comment}


}
{ \setcounter{ProblemNumber}{1}
  \restoregeometry
  \pagestyle{doc} }
%%%%End Exercises Definition

%%%%Begin Chapter Summary Definition
\newenvironment{summary}
{ 
	\newgeometry{margin=0.75in}	
	\pagestyle{exercise}
	\setcounter{ProblemNumber}{1}
	\noindent{\href{https://www.myopenmath.com/index.php}{\Huge\bfseries\color{blue!60!black} Exercises \thesection}}\\
	{\color{blue!60!black} \rule{1.1\textwidth}{3pt}}
	

%\begin{comment}
%\tikz[remember picture,overlay] {%
%\draw [blue!60!black,line width=10mm]
%(current page.south east)
%rectangle
%(current page.north east)
%;}
%\end{comment}


}
{ \setcounter{ProblemNumber}{1}
  \restoregeometry
  \pagestyle{doc} }

%%%%End Chapter Summary Definition

%%%% Graph Theory Vertex Label
\newcommand\extralabel[4][0mm]{\node[label={[label distance=#1]#3:#4}] at (#2){};}
% looks like \extralabel[distance(optional)]{node name}{angle}{label text}

% middle arrows on graphs
\tikzset{->-/.style={
 decoration={
   markings,
   mark=at position #1 with {\arrow{>}}
   },
   postaction={decorate}
  },
  ->-/.default=0.5 % set default value for arrow position
}

% circled numbers (or other text)
\newcommand*\circled[1]{\tikz[baseline=(char.base)]{
  \node[shape=circle,draw,inner sep=2pt] (char) {#1};}}