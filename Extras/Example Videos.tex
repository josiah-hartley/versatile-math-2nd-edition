\documentclass[8pt]{beamer}
\usepackage{tikz,pgfplots,etoolbox}
\usepackage[breakable,most]{tcolorbox}
\usepackage[T1]{fontenc}
\usefonttheme{serif}
\setbeamercolor{local structure}{fg=black}
%\input{PreCalcLayout}
\graphicspath{ {../Images/} }

\newcommand{\extitle}[1]{\frametitle{\fontfamily{fvs}\selectfont \small\color{black!70!blue!80!cyan}\uppercase{\bfseries Example: #1}}}

\newenvironment{exsol}
{
\begin{tcolorbox}[colframe=black!50!blue!50!cyan,
colback=white,
bottomrule=0mm,
rightrule=0mm,
sharp corners=all] 
%\Large\color{black!70!blue!80!cyan}\scshape #2}\\

}
{ \end{tcolorbox}}

\begin{document}

% Section 1.2
\begin{frame}
\extitle{Setting up Percentage Problems}
\begin{enumerate}[(a)]
\item What is 75\% of 690?
\item Forty is what percentage of 150?
\item Eight is 62\% of what?
\end{enumerate}

\begin{exsol}
\vspace{2.5in}
\text{}
\end{exsol}
\end{frame}

\begin{frame}
\extitle{Discount}
For months you have been wanting a 47'' LCD flat screen television, but the price has been too high. The store is having a one-day sale on all televisions in the store. For one day only you can take 25\% off any television. The regular price on the television
you want is \$1099.
\begin{enumerate}[(a)]
\item What is the sale price?
\item What will the final price, including sales tax, be if the sales tax rate is 8\%?
\end{enumerate}

\begin{exsol}
\vspace{2.5in}
\text{}
\end{exsol}
\end{frame}

% Section 1.3
\begin{frame}
\extitle{Using the TVM Solver}
Use the TVM solver on your calculator to find the present value needed if we want a future value of \$5000 in 6 years, if we can earn 4.3\% interest compounded monthly.

\begin{exsol}
\vspace{3in}
\text{}
\end{exsol}
\end{frame}

% Section 1.4
\begin{frame}
\extitle{Planning for Retirement}
Kevin is 30 years old, and he is preparing to begin saving for retirement.  He expects to retire at age 67, and for planning purposes, he assumes he'll live to age 95.  Based on cursory research, he expects that his investments can average a return of 7\% annually, and after retirement, he will move his money into more conservative investments returning 5\% annually.  In order to be able to withdraw \$3000 per month after retirement, how much should he plan to save each month?

\begin{exsol}
\vspace{3in}
\text{}
\end{exsol}
\end{frame}

\begin{frame}
\extitle{TVM Solver: Savings Annuity}
If you deposit \$250 each month into an IRA earning 7\% interest, how much will you have in the account after 35 years?  Use the TVM Solver on your graphing calculator.

\begin{exsol}
\vspace{3in}
\text{}
\end{exsol}
\end{frame}

\begin{frame}
\extitle{TVM Solver: Payout Annuity}
You expect to have \$500,000 in your IRA when you retire, and you want to be able to take monthly withdrawals for a total of 30 years.  If your account earns 8\% interest, how much will you be able to withdraw each month?  Use the TVM Solver on your graphing calculator.

\begin{exsol}
\vspace{3in}
\text{}
\end{exsol}
\end{frame}

% Section 1.5
\begin{frame}
\extitle{Buying a Condo}
The price of a condominium is \$180,000, and your bank offers a 30-year fixed mortgage at 4\% interest.  You have \$32,000 available right now.
\begin{enumerate}[(a)]
\item Your banker tells you to expect \$5000 in closing costs.  What percentage down payment can you afford?  Will you need mortgage insurance?
\item What will the principal be on the mortgage?
\item What will your monthly P\&I payment be?
\item In addition to principal and interest, your monthly payment will need to account for property taxes, homeowners insurance, and mortgage insurance, if necessary (find out in part (a)):
\begin{center}
\begin{tabular}{l r}
\textbf{Property Taxes:} & 1.5\% of the home value per year\\
\textbf{Homeowners Insurance:} & \$900 per year\\
\textbf{Mortgage Insurance:} & \$40 per month
\end{tabular}
\end{center}
What will your total monthly payment amount be?
\item How much will you pay in total over 30 years in principal and interest?
\item How much interest will you pay in total?
\end{enumerate}

\begin{exsol}
\vspace{3in}
\text{}
\end{exsol}
\end{frame}

\begin{frame}
\extitle{Changing the Interest Rate}
Compare a 30-year fixed-rate loan at 4\% for \$200,000 to the same loan at 3.5\% by finding the total amount paid in interest for both versions.

\begin{exsol}
\vspace{3in}
\text{}
\end{exsol}
\end{frame}

\begin{frame}
\extitle{Changing the Loan Amount}
Compare a 30-year fixed-rate loan at 4\% for \$200,000 to the same loan for \$180,000 by finding the total amount paid in interest for both versions.

\begin{exsol}
\vspace{3in}
\text{}
\end{exsol}
\end{frame}

\begin{frame}
\extitle{Changing the Length of the Loan}
Compare a 30-year fixed-rate loan at 4\% for \$200,000 to the same loan for 20 years and for 15 years by finding the total amount paid in interest for all three versions.

\begin{exsol}
\vspace{3in}
\text{}
\end{exsol}
\end{frame}

% Section 1.6
\begin{frame}
\extitle{Income Tax}
Using the tax table for 2020, how much would a married taxpayer who files separately from their spouse owe on a taxable income of \$98,400?

\begin{exsol}
\vspace{3in}
\text{}
\end{exsol}
\end{frame}

\begin{frame}
\extitle{Tax Calculation}
Use the 2020 tax table in the textbook to calculate the final tax owed by a single taxpayer whose details are given below.
\begin{center}
\begin{tabular}{r l}
Gross income: & \$65,000\\
Deductions: & \$3000: charitable donations\\
& \$6000: contribution to traditional IRA\\
& \$1500: education expenses\\
& \$300: cost of tax preparation\\
Tax credit: & \$500: energy-efficient appliances
\end{tabular}
\end{center}

\begin{exsol}
\vspace{3in}
\text{}
\end{exsol}
\end{frame}

% Section 2.2
\begin{frame}
\extitle{Plotting Points on a Calculator}
According to the U.S. Census Bureau, the number of Americans over the age of 100 is increasing.  The Census Bureau reported the following data, where the number of people is measured in the thousands:
\begin{center}
\begin{tabular}{c c}
\textbf{Year} & \textbf{Number}\\
& \textbf{(thousands)}\\
\hline
& \\
1994 & 50\\
1996 & 56\\
1998 & 65\\
2000 & 75\\
2002 & 94\\
2004 & 110
\end{tabular}
\end{center}

Graph this data using a graphing calculator.

\begin{exsol}
\vspace{3in}
\text{}
\end{exsol}
\end{frame}

\begin{frame}
\extitle{Fitting a Quadratic Model on a Calculator}
Using the same census data as in the previous example, find a quadratic model that can be used to predict how many Americans will be over the age of 100 in a given year.

\begin{exsol}
\vspace{3in}
\text{}
\end{exsol}
\end{frame}

\begin{frame}
\extitle{Making Predictions with a Quadratic Model}
A study designed to track the gas mileage of a car based on its speed found the following results.
\begin{center}
\begin{tabular}{c c}
\textbf{Speed (mph)} & \textbf{Mileage (mpg)}\\
\hline
& \\
15 & 22.3\\
20 & 25.5\\
25 & 27.5\\
30 & 29.0\\
35 & 28.8\\
40 & 30.0\\
45 & 29.9\\
50 & 30.2\\
55 & 30.4\\
60 & 28.8\\
65 & 27.4\\
70 & 25.3\\
75 & 23.3
\end{tabular}
\end{center}

\begin{enumerate}[(a)]
\item Use a graphing calculator to plot the data.
\item Find a quadratic model that best fits the data.
\item Based on this model, what gas mileage should be expected at 62 miles per hour?  At 90 miles per hour?  Which of these predictions is likely to be more reliable?
\item Based on the model, what speeds are likely to produce a mileage of 28 miles per gallon?
\end{enumerate}

\begin{exsol}
\vspace{3in}
\text{}
\end{exsol}
\end{frame}

% Section 2.3
\begin{frame}
\extitle{Solving for Time
In a previous example, we modeled the population growth of Frederick County from 2013 onward using the following equation:
\[P_t = 241,409(1+0.008)^t\]
Using this model, predict when the population will reach 400,000 people.

\begin{exsol}
\vspace{3in}
\text{}
\end{exsol}
\end{frame}

\begin{frame}
\extitle{Exponential Regression}
Build an exponential population model for the U.S. using data from 2005 to 2019.

\begin{exsol}
\vspace{3in}
\text{}
\end{exsol}
\end{frame}

% Section 2.4
\begin{frame}
\extitle{Logistic Regression}
Build a logistic population model for New York City using the population data below.
\begin{center}
\begin{tabular}{c c}
\textbf{Year} & \textbf{Population (millions)}\\
\hline
& \\
1900 & 3.44\\
1910 & 4.77\\
1920 & 5.62\\ 
1930 & 6.93\\
1940 & 7.45\\
1950 & 7.89\\
1960 & 7.78\\
1970 & 7.89\\
\end{tabular}
\end{center}

\begin{exsol}
\vspace{3in}
\text{}
\end{exsol}
\end{frame}

% Section 3.1
\begin{frame}
\extitle{Representative Samples}
Decide whether each of the following sampling methods is likely to produce a representative sample.

\begin{enumerate}[(a)]
\item To find the average annual income of all adults in the United States, sample representatives in the US Congress.
\item To find out the most popular cereal among children under the age of 10, stand outside a large supermarket one day and poll every twentieth child under the age of 10 who enters the supermarket.
\end{enumerate}

\begin{exsol}
\vspace{3in}
\text{}
\end{exsol}
\end{frame}

\begin{frame}
\extitle{Sampling Methods}
Determine the type of sampling used in each of the following scenarios.

\begin{exsol}
\begin{enumerate}[(a)]
\item A soccer coach selects six players from a group of boys aged eight to ten, seven players from a group of boys aged 11 to 12, and three players from a group of boys aged 13 to 14 to form a recreational soccer team.

\item A pollster interviews all human resource personnel in five different high tech companies.

\item A high school educational counselor interviews 50 female teachers and 50 male teachers.

\item A medical researcher interviews every third cancer patient from a list of cancer patients at a local hospital.

\item A high school counselor uses a computer to generate 50 random numbers and then picks students whose names correspond to the numbers.

\item A student interviews classmates in his algebra class to determine how many pairs of jeans a student at his school owns, on the average.
\end{enumerate}
\end{exsol}
\end{frame}

\begin{frame}
\extitle{Quiz Score Samples}
Use the random number generator on your calculator to generate different types of samples from the data below.  Find the average score for each sample and compare the results for each method.\\

This table displays six sets of quiz scores (out of 10 points) for an elementary statistics class.
\begin{center}
\begin{tabular}{l l l l l l}
A & B & C & D & E & F\\
\hline
& & & & & \\
5 & 7 & 10 & 9 & 8 & 3\\
10 & 5 & 9 & 8 & 7 & 6\\
9 & 10 & 8 & 6 & 7 & 9\\
9 & 10 & 10 & 9 & 8 & 9\\
7 & 8 & 9 & 5 & 7 & 4\\
9 & 9 & 9 & 10 & 8 & 7\\
7 & 7 & 10 & 9 & 8 & 8\\
8 & 8 & 9 & 10 & 8 & 8\\
9 & 7 & 8 & 7 & 7 & 8\\
8 & 8 & 10 & 9 & 8 & 7\\
\end{tabular}
\end{center}

\begin{exsol}
\vspace{3in}
\text{}
\end{exsol}
\end{frame}

% Section 3.2
\begin{frame}
\extitle{Dot Plot}
Draw a dot plot to summarize the following data, the ages of 30 randomly chosen NBA players:
\[22, 28, 20, 24, 26, 21, 27, 28, 31, 29, 24, 22, 21, 25, 22,\]
\[25, 30, 29, 20, 22, 36, 24, 23, 36, 24, 29, 21, 21, 26, 23\]

\begin{exsol}
\vspace{3in}
\text{}
\end{exsol}
\end{frame}

\begin{frame}
\extitle{Categorical Frequency Distribution}
Create a frequency table for the players' positions from the NBA dataset, including a relative frequency column.

\begin{exsol}
\vspace{3in}
\text{}
\end{exsol}
\end{frame}

\begin{frame}
\extitle{Grouped Frequency Distribution}
Build a grouped frequency table for points per game for the NBA players dataset, using a class width of 5.

\begin{exsol}
\vspace{3in}
\text{}
\end{exsol}
\end{frame}

\begin{frame}
\extitle{Histogram}
Build a histogram for points per game in the NBA dataset, using grouped classes with a class width of 5.

\begin{exsol}
\vspace{3in}
\text{}
\end{exsol}
\end{frame}

\begin{frame}
\extitle{Bar Chart}
Build a bar chart for the players' positions in the NBA dataset.

\begin{exsol}
\vspace{3in}
\text{}
\end{exsol}
\end{frame}

\begin{frame}
\extitle{Stem and Leaf Plot}
Suppose you gathered data on how long it took you to get ready in the morning.  For 40 days, you measured the amount of time between when your alarm went off and when you left the house.  The results are below, rounded to the nearest minute:
\begin{center}
\begin{tabular}{c c c c c c c c c c}
35 & 28 & 25 & 23 & 23 & 32 & 29 & 19 & 21 & 13\\
24 & 26 & 25 & 31 & 30 & 20 & 25 & 29 & 37 & 26\\
32 & 36 & 18 & 17 & 15 & 24 & 21 & 16 & 19 & 30\\
38 & 27 & 22 & 24 & 28 & 17 & 31 & 32 & 21 & 28\\
\end{tabular}
\end{center}
Build a stem-and-leaf plot for this data.

\begin{exsol}
\vspace{3in}
\text{}
\end{exsol}
\end{frame}

\begin{frame}
\extitle{Scatterplot: TV Price}
The following table shows, for a sample of Samsung LCD TVs, their size and their price.  Construct a scatterplot for this data.
\begin{center}
\begin{tabular}{c c | c c}
Size (in.) & Price (\$) & Size (in.) & Price (\$)\\
\hline
\\
43 & 500 & 60 & 1200\\
55 & 900 & 45 & 1600\\
51 & 900 & 19 & 200\\
32 & 400 & 55 & 2200\\
51 & 1200 & 60 & 1700\\
37 & 500 & 55 & 2000\\
60 & 2800 & 22 & 300\\
60 & 1100 & 40 & 600\\
46 & 1600 & 40 & 900
\end{tabular}
\end{center}

\begin{exsol}
\vspace{3in}
\text{}
\end{exsol}
\end{frame}

% Section 3.3
\begin{frame}
\extitle{Average NBA Salary}
Find the average salary of the players listed in the NBA dataset.

\begin{exsol}
\vspace{3in}
\text{}
\end{exsol}
\end{frame}

\begin{frame}
\extitle{Median NBA Salary}
Find the median of the salaries listed in the NBA dataset.

\begin{exsol}
\vspace{3in}
\text{}
\end{exsol}
\end{frame}

\begin{frame}
\extitle{Weighted Average}
Find the final score of the student whose grades are listed below, using both the points system and the percentage system for defining weights.
\begin{center}
\begin{tabular}{l l l l}
\textbf{Assignment} & \textbf{Score} & \textbf{Weight} & \textbf{Points}\\
\hline
& \\
Test 1 & 85\% & 20\% & 200\\
Test 2 & 92\% & 20\% & 200\\
Test 3 & 87\% & 20\% & 200\\
Homework & 95\% & 15\% & 150\\
Project & 92\% & 10\% & 100\\
Final Exam & 91\% & 15\% & 150
\end{tabular}
\end{center}

\begin{exsol}
\vspace{3in}
\text{}
\end{exsol}
\end{frame}

\begin{frame}
\extitle{Mode}
Find the mode of the dataset summarized below, the ages of players in the NBA dataset.
\begin{center}
\begin{tabular}{c c}
\textbf{Age} & \textbf{Frequency}\\
\hline
& \\
20 & 2\\
21 & 4\\
22 & 4\\
23 & 2\\
24 & 4\\
25 & 2\\
26 & 2\\
27 & 1\\
28 & 2\\
29 & 3\\
30 & 1\\
31 & 1\\
36 & 2
\end{tabular}
\end{center}

\begin{exsol}
\vspace{3in}
\text{}
\end{exsol}
\end{frame}

\begin{frame}
\extitle{Range of NBA Players' Heights}
Find the range of heights for the players listed in the NBA dataset.

\begin{exsol}
\vspace{3in}
\text{}
\end{exsol}
\end{frame}

% Section 3.4 -- PICK UP HERE
\begin{frame}
\extitle{Exponential Regression}
Build an exponential population model for the U.S. using data from 2005 to 2019.

\begin{exsol}
\vspace{3in}
\text{}
\end{exsol}
\end{frame}

\begin{frame}
\extitle{Exponential Regression}
Build an exponential population model for the U.S. using data from 2005 to 2019.

\begin{exsol}
\vspace{3in}
\text{}
\end{exsol}
\end{frame}

\begin{frame}
\extitle{Exponential Regression}
Build an exponential population model for the U.S. using data from 2005 to 2019.

\begin{exsol}
\vspace{3in}
\text{}
\end{exsol}
\end{frame}

\begin{frame}
\extitle{Exponential Regression}
Build an exponential population model for the U.S. using data from 2005 to 2019.

\begin{exsol}
\vspace{3in}
\text{}
\end{exsol}
\end{frame}

\begin{frame}
\extitle{Exponential Regression}
Build an exponential population model for the U.S. using data from 2005 to 2019.

\begin{exsol}
\vspace{3in}
\text{}
\end{exsol}
\end{frame}

% Section 3.5
\begin{frame}
\extitle{Exponential Regression}
Build an exponential population model for the U.S. using data from 2005 to 2019.

\begin{exsol}
\vspace{3in}
\text{}
\end{exsol}
\end{frame}

\end{document}