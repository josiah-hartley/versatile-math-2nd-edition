\begin{exercises}
\ptwo{Nineteen immigrants to the U.S. were asked how many years, to the nearest year, they have lived in the U.S.  The data are as follows: 
\begin{align*}
&2,\ 5,\ 7,\ 2,\ 2,\ 10,\ 20,\ 15,\ 0,\\ 
7,\ &0,\ 20,\ 5,\ 12,\ 15,\ 12,\ 4,\ 5,\ 10
\end{align*} 
Draw a dot plot to summarize this data.}
\ptwo{A group of students earned the following final grades:
\begin{center}
B, C, A, B, B, D, C, C, C, F, A, C, B, B, B, C, B, D
\end{center}
Draw a dot plot to summarize this data.}

\ptwo{A store tracked how many iPads were sold each day for fifty days, and their data is below.
\begin{center}
\begin{tabular}{c c c c c c c c c c}
4 & 2 & 3 & 2 & 5 & 5 & 1 & 3 & 3 & 2\\
3 & 2 & 2 & 3 & 2 & 2 & 2 & 3 & 0 & 1\\
3 & 1 & 1 & 5 & 4 & 1 & 2 & 4 & 3 & 5\\
2 & 0 & 0 & 3 & 2 & 3 & 3 & 3 & 2 & 2\\
0 & 4 & 2 & 4 & 3 & 1 & 1 & 4 & 0 & 1
\end{tabular}
\end{center}
Construct a frequency table (including a relative frequency column) to describe this data.}
\ptwo{Twenty students were asked how many hours they worked per day.  Their responses are as follows:
\begin{center}
\begin{tabular}{c c c c}
5 & 6 & 3 & 3\\
2 & 4 & 7 & 5\\
2 & 3 & 5 & 6\\
5 & 4 & 4 & 3\\
5 & 2 & 5 & 3
\end{tabular}
\end{center}
Construct a frequency table (including a relative frequency column) to describe this data.}

\pone{Fifty part-time students were asked how many courses they were taking this semester. The (incomplete) results are shown below. Fill in the blank cells to complete the table.
\begin{center}
\begin{tabular}{c c c }
\textbf{Number of Courses} & \textbf{Frequency} & \textbf{Relative Frequency}\\
\hline
& & \\
1 & 30 & 0.6\\
2 & 15 &    \\
3 &    &    
\end{tabular}
\end{center}
}

\ptwo{A group of 20 students were polled and asked what year they belonged to, whether they were freshmen (FR), sophomores (SO), juniors (JR), or seniors (SR).  The results are written below.\begin{center}
\begin{tabular}{c c c c}
FR & JR & SO & JR\\
SR & FR & SO & SO\\
SO & SR & SO & SR\\
SR & FR & SR & SO\\
SR & SO & JR & JR
\end{tabular}
\end{center}
Construct a frequency table (including a relative frequency column) to describe this data.}
\ptwo{A group of 20 registered voters were polled and asked what party they belonged to, whether they were Republicans (R), Democrats (D), Green Party members (G), or independent (I).  The results are written below.\begin{center}
\begin{tabular}{c c c c}
R & R & D & D\\
G & D & R & D\\
I & R & R & D\\
I & D & R & I\\
R & D & R & D
\end{tabular}
\end{center}
Construct a frequency table (including a relative frequency column) to describe this data.}

\ptwo{The following is the average daily temperature for Frederick, Maryland for the month of June: \begin{align*}74, 60, 58, 58, 64, 67, 64, 74, 72, 70,\\ 78, 80, 80, 79, 80, 80, 70, 83, 76, 78,\\ 81, 78, 81, 70, 70, 71, 66, 66, 68, 74.\end{align*}
\begin{enumerate}[(a)]
\item Construct a grouped frequency and relative frequency distribution using a class width of 5, starting at 55.
\item Construct a histogram from the frequency distribution. 
\end{enumerate}
}
\ptwo{A researcher gathered data on hours of video games played by school-aged children and young adults. She collected the following data: \begin{align*}0, 0, 1, 1, 1, 2, 2, 3, 3, 3,\\ 4, 4, 4, 4, 5, 5, 5, 6, 6, 7,\\ 7, 7, 8, 8, 8, 8, 8, 9, 9, 9,\\ 10, 10, 11, 12, 12, 12, 12, 13. \end{align*}
\begin{enumerate}[(a)]
\item Construct a grouped frequency and relative frequency distribution using a class width of 2, starting at 0.
\item Construct a histogram from the frequency distribution. 
\end{enumerate}}
\pagebreak

\textit{For exercises 10--13, use the frequency table below, which contains the total number of deaths worldwide as a result of earthquakes for the period from 2000 to 2012.}\\
\begin{center}
\begin{tabular}{c c}
\textbf{Year} & \textbf{Total Number of Deaths}\\
\hline
& \\
2000 & 231\\
2001 & 21,357\\
2002 & 11,685\\
2003 & 33,819\\
2004 & 228,802\\
2005 & 88,003\\
2006 & 6,605\\
2007 & 712\\
2008 & 88,011\\
2009 & 1,790\\
2010 & 320,120\\
2011 & 21,953\\
2012 & 768\\
\textbf{Total} & \textbf{823,356}
\end{tabular}
\end{center}

\ptwo{What is the frequency of deaths measured from 2006 through 2009?}
\ptwo{What percentage of deaths occurred after 2009 (from 2010 onwards)?}

\ptwo{What is the relative frequency of deaths that occurred in 2003 or earlier?}
\ptwo{What is the percentage of deaths that occurred in 2004?}

\pone{What is wrong with the following grouped frequency distribution?
\begin{center}
\begin{tabular}{c | c}
\textbf{Grades} & \textbf{Frequency}\\
\hline
50--55 & 2\\
55--60 & 4\\
60--70 & 9\\
70--80 & 15\\
80--90 & 7\\
90 and above & 4
\end{tabular}
\end{center}
\begin{enumerate}[(a)]
\item The classes do not all have the same width.
\item The classes overlap.
\item There are open-ended classes.
\item All of the above.
\end{enumerate}}

\ptwo{Draw a bar chart for the dataset in problem 6.}
\ptwo{Draw a bar chart for the dataset in problem 7.}

\ptwo{The scores for a math test are shown below, ordered from smallest to largest.
\begin{center}
\begin{tabular}{c c c c c c}
42 & 49 & 49 & 53 & 55 & 55\\
61 & 63 & 67 & 68 & 68 & 69\\
69 & 72 & 73 & 74 & 78 & 80\\
83 & 88 & 88 & 88 & 90 & 92\\
94 & 94 & 94 & 95 & 96 & 100
\end{tabular}
\end{center}
Build a stem-and-leaf plot for this data.}
\ptwo{A basketball team's scores for the last 30 games are shown below, ordered from smallest to largest.
\begin{center}
\begin{tabular}{c c c c c c}
32 & 32 & 33 & 34 & 38 & 40\\
42 & 42 & 43 & 44 & 46 & 47\\
47 & 48 & 48 & 48 & 49 & 50\\
50 & 51 & 52 & 52 & 52 & 53\\
54 & 56 & 57 & 57 & 60 & 61
\end{tabular}
\end{center}
Build a stem-and-leaf plot for this data.}
\vfill
\pagebreak

\pone{
The following stem-and-leaf plots compare the ages of 30 actors and 30 actresses at the time they won the Oscar award for Best Actor or Actress.
\begin{center}
\begin{tabular}{|r | c | l|}
\hline
Actors & Stems & Actresses\\
\hline
& 2 & 146667\\
\hline
98753221 & 3 & 00113344455778\\
\hline
88776543322100 & 4 & 11129\\
\hline
6651 & 5 & \\
\hline
210 & 6 & 011\\
\hline
6 & 7 & 4\\
\hline
& 8 & 0\\
\hline
\end{tabular}
\end{center}
\begin{enumerate}[(a)]
\item What is the age of the youngest actor to win an Oscar?
\item What is the age difference between the oldest and the youngest actress to win an Oscar?
\item What is the oldest age shared by two actors to win an Oscar?
\end{enumerate}
}

\ptwo{The table below shows the yearly tuition of 8 universities, as well as the average mid-career salaries for graduates of each university.
\begin{center}
\begin{tabular}{l c c}
\textbf{University} & \textbf{Tuition (\$)} & \textbf{Salary (\$)}\\
\hline
& & \\
Princeton & 28,540 & 137,000\\
Harvey Mudd & 40,133 & 135,000\\
CalTech & 39,900 & 127,000\\
MIT & 42,050 & 118,000\\
Lehigh University & 43,220 & 118,000\\
NYU-Poly & 39,565 & 117,000\\
Babson College & 40,400 & 117,000\\
Stanford & 54,506 & 114,000
\end{tabular}
\end{center}
Draw a scatterplot for this data, using $x$ to represent tuition and $y$ to represent salary.}
\ptwo{The table below shows the frequency of chirps for the striped ground cricket compared to the ambient temperature.
\begin{center}
\begin{tabular}{c c}
\textbf{Chirps per Second} & \textbf{Temperature ($^\circ$F)}\\
\hline
& \\
20.0 & 88.6\\
16.0 & 71.6\\
19.8 & 93.3\\
18.4 & 84.3\\
17.1 & 80.6\\
15.5 & 75.2\\
14.7 & 69.7\\
17.1 & 82.0\\
15.4 & 69.4
\end{tabular}
\end{center}
Draw a scatterplot for this data, using $x$ to represent the chirping frequency and $y$ to represent temperature.}

\end{exercises}