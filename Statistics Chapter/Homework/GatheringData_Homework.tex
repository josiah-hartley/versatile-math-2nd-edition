\begin{exercises}
\emph{For problems 1--2, identify the population and sample.}\\
\ptwo{A political scientist surveys 28 of the current 106 representatives in a state's congress. Of them, 14 said they were supporting a new education bill, 12 said they were not supporting the bill, and 2 were undecided.}
\ptwo{The city of Frederick has 9500 registered voters. There are two candidates for the city council in an upcoming election: Marfani and Rahman. The day before the election, a telephone poll of 350 randomly selected registered voters was conducted. Of them, 112 said they would vote for Marfani, 207 said they would vote for Rahman, and 31 were undecided.}\\ 

\emph{For problems 3--4, decide whether the sampling method described is likely to produce a representative sample, and why or why not.}\\
\ptwo{A marriage counselor is interested in the studying divorce rates, so she gives her clients a survey.}
\ptwo{A fitness center wants to see how much use their treadmills get, so they pick random times during the day and record how many treadmills are in use each time.}

\emph{For problems 5--10, determine the type of sampling used in the given scenario.}\\
\ptwo{A high school principal polls 50 freshmen, 50 sophomores, 50 juniors, and 50 seniors regarding policy changes for after school activities.}
\ptwo{To check their accuracy, the Census Bureau draws a sample of several city blocks and recounts everyone in those blocks.}

\ptwo{A pollster walks around a busy shopping mall and asks people passing by how often they shop at the mall.}
\ptwo{Police at a DUI checkpoint stop every tenth car to check whether the driver is sober.}

\ptwo{A restaurant samples 100 sales from the past week by numbering all their receipts, generating 100 random numbers, and picking the receipts that correspond to those numbers.}
\ptwo{The provost at a university wants to know how a particular policy is affecting faculty, so she randomly selects 3 members of each department to survey.}

\end{exercises}