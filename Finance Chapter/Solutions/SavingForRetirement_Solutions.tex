\emph{In problems 1--3, a periodic deposit is made into an annuity with the given terms.  Find how much the annuity will hold at the end of the specified amount of time.}

\begin{enumerate}
\item \begin{tabular}{r l}
Regular deposit & \$250\\
Interest rate & 4\%\\
Frequency & Monthly\\
Time & 15 years\\
Future value & ?
\end{tabular} \answer{\$61,522.62}
\[F=\dfrac{PMT\left[\left(1+\dfrac{r}{n}\right)^{nt}-1\right]}{\left(\dfrac{r}{n}\right)} = \dfrac{250\left[\left(1+\dfrac{0.04}{12}\right)^{(12)(15)}-1\right]}{\left(\dfrac{0.04}{12}\right)} = 61,522.62\]

\item \begin{tabular}{r l}
Regular deposit & \$10\\
Interest rate & 5\%\\
Frequency & Daily\\
Time & 12 years\\
Future value & ?
\end{tabular} \answer{\$60,009.21}
\[F=\dfrac{PMT\left[\left(1+\dfrac{r}{n}\right)^{nt}-1\right]}{\left(\dfrac{r}{n}\right)} = \dfrac{10\left[\left(1+\dfrac{0.05}{365}\right)^{(365)(12)}-1\right]}{\left(\dfrac{0.05}{365}\right)} = 60,009.21\]

\item \begin{tabular}{r l}
Regular deposit & \$2000\\
Interest rate & 3\%\\
Frequency & Yearly\\
Time & 22 years\\
Future value & ?
\end{tabular} \answer{\$61,073.56}
\[F=\dfrac{PMT\left[\left(1+\dfrac{r}{n}\right)^{nt}-1\right]}{\left(\dfrac{r}{n}\right)} = \dfrac{2000\left[\left(1+\dfrac{0.03}{1}\right)^{(1)(22)}-1\right]}{\left(\dfrac{0.03}{1}\right)} = 61,073.56\]
\end{enumerate}

\emph{In problems 4--6, find how much should be regularly deposited into an annuity with the given terms in order to have the specified final amount in the account.}

\begin{enumerate}
\setcounter{enumi}{3}
\item \begin{tabular}{r l}
Regular deposit & ?\\
Interest rate & 5\%\\
Frequency & Monthly\\
Time & 18 years\\
Future value & \$50,000
\end{tabular} \answer{\$143.18}
\[PMT = \dfrac{F\left(\dfrac{r}{n}\right)}{\left[\left(1+\dfrac{r}{n}\right)^{nt}-1\right]} = \dfrac{50,000\left(\dfrac{0.05}{12}\right)}{\left[\left(1+\dfrac{0.05}{12}\right)^{(12)(18)}-1\right]} = 143.18\]

\item \begin{tabular}{r l}
Regular deposit & ?\\
Interest rate & 6\%\\
Frequency & Weekly\\
Time & 10 years\\
Future value & \$27,000
\end{tabular} \answer{\$37.92}
\[PMT = \dfrac{F\left(\dfrac{r}{n}\right)}{\left[\left(1+\dfrac{r}{n}\right)^{nt}-1\right]} = \dfrac{27,000\left(\dfrac{0.06}{52}\right)}{\left[\left(1+\dfrac{0.06}{52}\right)^{(52)(10)}-1\right]} = 37.92\]

\item \begin{tabular}{r l}
Regular deposit & ?\\
Interest rate & 3.5\%\\
Frequency & Yearly\\
Time & 35 years\\
Future value & \$200,000
\end{tabular} \answer{\$2999.67}
\[PMT = \dfrac{F\left(\dfrac{r}{n}\right)}{\left[\left(1+\dfrac{r}{n}\right)^{nt}-1\right]} = \dfrac{200,000\left(\dfrac{0.035}{1}\right)}{\left[\left(1+\dfrac{0.035}{1}\right)^{(1)(35)}-1\right]} = 2999.67\]
\end{enumerate}

\emph{In problems 7--9, you want to be able to withdraw the specified amount periodically from a payout annuity with the given terms.  Find how much the account needs to hold to make this possible.}

\begin{enumerate}
\setcounter{enumi}{6}
\item \begin{tabular}{r l}
Regular withdrawal & \$1000\\
Interest rate & 5\%\\
Frequency & Monthly\\
Time & 20 years\\
Account balance & ?
\end{tabular} \answer{\$151,525.31}
\[P = \dfrac{PMT\left[1-\left(1+\dfrac{r}{n}\right)^{-nt}\right]}{\left(\dfrac{r}{n}\right)} = \dfrac{1000\left[1-\left(1+\dfrac{0.05}{12}\right)^{-(12)(20)}\right]}{\left(\dfrac{0.05}{12}\right)} = 151,525.31\]

\item \begin{tabular}{r l}
Regular withdrawal & \$200\\
Interest rate & 3\%\\
Frequency & Weekly\\
Time & 15 years\\
Account balance & ?
\end{tabular} \answer{\$125,593.56}
\[P = \dfrac{PMT\left[1-\left(1+\dfrac{r}{n}\right)^{-nt}\right]}{\left(\dfrac{r}{n}\right)} = \dfrac{200\left[1-\left(1+\dfrac{0.03}{52}\right)^{-(52)(15)}\right]}{\left(\dfrac{0.03}{52}\right)} = 125,593.56\]

\item \begin{tabular}{r l}
Regular withdrawal & \$20,000\\
Interest rate & 5.5\%\\
Frequency & Yearly\\
Time & 25 years\\
Account balance & ?
\end{tabular} \answer{\$268,278.65}
\[P = \dfrac{PMT\left[1-\left(1+\dfrac{r}{n}\right)^{-nt}\right]}{\left(\dfrac{r}{n}\right)} = \dfrac{20,000\left[1-\left(1+\dfrac{0.055}{1}\right)^{-(1)(25)}\right]}{\left(\dfrac{0.055}{1}\right)} = 268,278.65\]
\end{enumerate}

\emph{In problems 10--12, you expect to have the given amount in an account with the given terms.  Find how much you can withdraw periodically in order to make the account last the specified amount of time.}

\begin{enumerate}
\setcounter{enumi}{9}
\item \begin{tabular}{r l}
Regular withdrawal & ?\\
Interest rate & 4\%\\
Frequency & Monthly\\
Time & 18 years\\
Account balance & \$300,000
\end{tabular} \answer{\$1950.59}
\[PMT = \dfrac{P\left(\dfrac{r}{n}\right)}{1-\left(1+\dfrac{r}{n}\right)^{-nt}} = \dfrac{300,000\left(\dfrac{0.04}{12}\right)}{1-\left(1+\dfrac{0.04}{12}\right)^{-(12)(18)}} = 1950.59\]

\item \begin{tabular}{r l}
Regular withdrawal & ?\\
Interest rate & 5\%\\
Frequency & Weekly\\
Time & 20 years\\
Account balance & \$250,000
\end{tabular} \answer{\$380.29}
\[PMT = \dfrac{P\left(\dfrac{r}{n}\right)}{1-\left(1+\dfrac{r}{n}\right)^{-nt}} = \dfrac{250,000\left(\dfrac{0.05}{52}\right)}{1-\left(1+\dfrac{0.05}{52}\right)^{-(52)(20)}} = 380.29\]

\item \begin{tabular}{r l}
Regular withdrawal & ?\\
Interest rate & 2.85\%\\
Frequency & Monthly\\
Time & 30 years\\
Account balance & \$1,000,000
\end{tabular} \answer{\$4135.57}
\[PMT = \dfrac{P\left(\dfrac{r}{n}\right)}{1-\left(1+\dfrac{r}{n}\right)^{-nt}} = \dfrac{1,000,000\left(\dfrac{0.0285}{12}\right)}{1-\left(1+\dfrac{0.0285}{12}\right)^{-(12)(30)}} = 4135.57\]

\item You deposit \$200 each month into an account earning 3\% interest compounded monthly.
\begin{enumerate}[(a)]
\item How much will you have in the account in 30 years? \answersub{\$116,547.38}
\[F=\dfrac{PMT\left[\left(1+\dfrac{r}{n}\right)^{nt}-1\right]}{\left(\dfrac{r}{n}\right)} = \dfrac{200\left[\left(1+\dfrac{0.03}{12}\right)^{(12)(30)}-1\right]}{\left(\dfrac{0.03}{12}\right)} = 116,547.38\]

\item How much total money will you put into the account? \answersub{\$72,000}
\[(200)(12)(30) = 72,000\]

\item How much total interest will you earn? \answersub{\$44,547.38}
\[116,547.38 - 72,000 = 44,547.38\]
\end{enumerate}
\pagebreak

\item You deposit \$1000 each year into an account earning 8\% interest compounded annually.
\begin{enumerate}[(a)]
\item How much will you have in the account in 10 years? \answersub{\$14,486.56}
\[F=\dfrac{PMT\left[\left(1+\dfrac{r}{n}\right)^{nt}-1\right]}{\left(\dfrac{r}{n}\right)} = \dfrac{1000\left[\left(1+\dfrac{0.08}{1}\right)^{(1)(10)}-1\right]}{\left(\dfrac{0.08}{1}\right)} = 14,486.56\]

\item How much total money will you put into the account? \answersub{\$10,000}
\[(1000)(1)(10) = 10,000\]

\item How much total interest will you earn? \answersub{\$4,486.56}
\[14,486.56 - 10,000 = 4,486.56\]
\end{enumerate}

\item Evelyn has \$500,000 saved for retirement in an account earning 6\% interest, compounded monthly.  How much will she be able to withdraw each month if she wants to take withdrawals for 20 years? \answer{\$3582.16}
\[PMT = \dfrac{P\left(\dfrac{r}{n}\right)}{1-\left(1+\dfrac{r}{n}\right)^{-nt}} = \dfrac{500,000\left(\dfrac{0.06}{12}\right)}{1-\left(1+\dfrac{0.06}{12}\right)^{-(12)(20)}} = 3582.16\]

\item Luke already knows that he will have \$750,000 when he retires.  If he sets up a payout annuity for 30 years in an account paying 7\% interest, how much could the annuity provide each month? \answer{\$5814.74}
\[PMT = \dfrac{P\left(\dfrac{r}{n}\right)}{1-\left(1+\dfrac{r}{n}\right)^{-nt}} = \dfrac{750,000\left(\dfrac{0.07}{12}\right)}{1-\left(1+\dfrac{0.07}{12}\right)^{-(12)(30)}} = 4989.77\]

\item Michael is planning for retirement, and he estimates that he'll want to be able to withdraw \$2500 each month for 30 years once he retires.  He opens a Roth IRA and finds investments that he expects to return 5\% interest compounded monthly.
\begin{enumerate}[(a)]
\item How much will he need to have in the account when he retires in order to meet his goal? \answersub{\$465,704.04}
\[P = \dfrac{PMT\left[1-\left(1+\dfrac{r}{n}\right)^{-nt}\right]}{\left(\dfrac{r}{n}\right)} = \dfrac{2500\left[1-\left(1+\dfrac{0.05}{12}\right)^{-(12)(30)}\right]}{\left(\dfrac{0.05}{12}\right)} = 465,704.04\]

\item How much will he have to deposit each month for the next 40 years in order to get this balance at retirement? \answersub{\$305.18}
\[PMT = \dfrac{P\left(\dfrac{r}{n}\right)}{1-\left(1+\dfrac{r}{n}\right)^{-nt}} = \dfrac{465,704.04\left(\dfrac{0.07}{12}\right)}{1-\left(1+\dfrac{0.07}{12}\right)^{-(12)(40)}} = 305.18\]

\item How much interest will his deposits earn? \answersub{\$319,217.64}
\[465,704.04 - (305.18)(12)(40) = 319,217.64\]
\end{enumerate}

\item Rachel is planning for retirement, and she estimates that she'll want to be able to withdraw \$1800 each month for 25 years once she retires.  She opens a Roth IRA and finds investments that she expects to return 3.75\% interest compounded monthly.
\begin{enumerate}[(a)]
\item How much will she need to have in the account when she retires in order to meet her goal? \answersub{\$350,105.19}
\[P = \dfrac{PMT\left[1-\left(1+\dfrac{r}{n}\right)^{-nt}\right]}{\left(\dfrac{r}{n}\right)} = \dfrac{1800\left[1-\left(1+\dfrac{0.0375}{12}\right)^{-(12)(25)}\right]}{\left(\dfrac{0.0375}{12}\right)} = 350,105.19\]

\item How much will she have to deposit each month for the next 40 years in order to get this balance at retirement? \answersub{\$315.19}
\[PMT = \dfrac{P\left(\dfrac{r}{n}\right)}{1-\left(1+\dfrac{r}{n}\right)^{-nt}} = \dfrac{350,105.19\left(\dfrac{0.0375}{12}\right)}{1-\left(1+\dfrac{0.0375}{12}\right)^{-(12)(40)}} = 315.19\]

\item How much interest will her deposits earn? \answersub{\$198,816.05}
\[350,105.19 - (315.19)(12)(40) = 198,816.05\]
\end{enumerate}

\item Faith is 27 years old, she plans to retire at age 65, and she expects to live to age 92.  She expects that her investments can earn an average return of 8\% until retirement, and after retirement, she plans to earn 4\%.  If she wants to be able to withdraw \$2500 per month after retirement, how much should she start saving each month? \answer{\$167.50}
\begin{center}
Balance at retirement: $P = \dfrac{2500\left[1-\left(1+\dfrac{0.04}{12}\right)^{-(12)(27)}\right]}{\left(\dfrac{0.04}{12}\right)} = 494,845.51$\\
Deposits before retiring: $PMT = \dfrac{494,845.51\left(\dfrac{0.08}{12}\right)}{1-\left(1+\dfrac{0.08}{12}\right)^{-(12)(38)}} = 167.50$
\end{center}

\item Caleb is 30 years old, he plans to retire at age 68, and he expects to live to age 90.  He expects that his investments can earn an average return of 6\% until retirement, and after retirement, he plans to earn 5\%.  If he wants to be able to withdraw \$2000 per month after retirement, how much should he start saving each month? \answer{\$183.38}
\begin{center}
Balance at retirement: $P = \dfrac{2000\left[1-\left(1+\dfrac{0.05}{12}\right)^{-(12)(22)}\right]}{\left(\dfrac{0.05}{12}\right)} = 319,856.32$\\
Deposits before retiring: $PMT = \dfrac{319,856.32\left(\dfrac{0.6}{12}\right)}{1-\left(1+\dfrac{0.06}{12}\right)^{-(12)(38)}} = 183.38$
\end{center}
\end{enumerate}