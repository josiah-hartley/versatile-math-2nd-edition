\begin{enumerate}
\item A \line(1,0){75} \ is a tax at a consistent rate. \answer{flat tax}

\item A \line(1,0){75} \ is a tax for which the rate increases for higher taxed amounts \answer{progressive tax}

\item A \line(1,0){75} \ is a tax for which the rate decreases for higher taxed amounts \answer{regressive tax}

\item If the property taxes are \$1800 on a home valued at \$140,000, what is the effective tax rate? \answer{1.29\%}
\[\dfrac{1800}{140,000} = 0.0129 = 1.29\%\]

\item In state A, the gas tax is 28 cents per gallon, where the average pre-tax cost of gas is \$2.58 per gallon.  In state B, the gas tax is 25 cents per gallon, where the average pre-tax cost of gas is \$2.50.  Which state has a lower gas tax rate? \answer{State A}
\begin{center}
State A: $\dfrac{0.28}{2.58} = 0.1085 = 10.85\%$\\ \text{}\\
State B: $\dfrac{0.25}{2.50} = 0.1 = 10\%$
\end{center}

\item If the sales tax is \$16.05 on a purchase of \$214, what is the sales tax rate? \answer{7.5\%}
\[\dfrac{16.05}{214} = 0.075 = 7.5\%\]

\item Using the 2020 tax table given in the textbook, how much would a single taxpayer owe on a taxable income of \$55,000? \answer{\$7890}
\[(9875)(0.10) + (40,125-9875)(0.12) + (55,000-40,125)(0.22) = 7890\]

\item Using the 2020 tax table given in the textbook, how much would a married couple filing jointly owe on a taxable income of \$92,000? \answer{\$11,820}
\[(19,750)(0.10) + (80,250-19,750)(0.12) + (92,000-80,250)(0.22) = 11,820\]
\end{enumerate}
\pagebreak

\emph{For problems 9--14, use the 2020 tax table given in the textbook to calculate the tax owed by each taxpayer.}

\begin{enumerate}
\setcounter{enumi}{8}
\item \text{}\\
\begin{tabular}{r l}
Taxpayer: & Single\\
Gross income: & \$75,000\\
Deductions: & \$18,000: mortgage interest\\
& \$2500: property taxes\\
& \$2000: charitable donations\\
& \$300: cost of tax preparation\\
Tax credit: & \$800
\end{tabular} \answer{\$6870}
\begin{center}
Deductions: $18,000 + 2500 + 200 + 300 = 21,000$ (greater than standard)\\
Taxable income: $75,000 - 21,000 = 54,000$\\
Initial tax: $(9875)(0.10) + (40,125-9875)(0.12) + (54,000-40,125)(0.22) = 7670$\\
Apply credit: $7670 - 800 = 6870$
\end{center}

\item \text{}\\
\begin{tabular}{r l}
Taxpayer: & Single\\
Gross income: & \$40,000\\
Deductions: & \$10,000: mortgage interest\\
& \$2000: property taxes\\
& \$300: charitable donations\\
Tax credit: & \$1300
\end{tabular} \answer{\$1814.50}
\begin{center}
Deductions: $10,000 + 2000 + 300 = 12,300$ (lower than standard; use standard)\\
Taxable income: $40,000 - 12,400 = 27,600$\\
Initial tax: $(9875)(0.10) + (27,600-9875)(0.12) = 3114.50$\\
Apply credit: $3114.50 - 1300 = 1814.50$
\end{center}

\item \text{}\\
\begin{tabular}{r l}
Taxpayer: & Married, filing jointly\\
Gross income: & \$85,500\\
Deductions: & \$5000: charitable donations\\
& \$3750: state taxes\\
Tax credit: & \$750
\end{tabular} \answer{\$6139}
\begin{center}
Deductions: $5000 + 3750 = 8750$ (lower than standard; use standard)\\
Taxable income: $85,500 - 24,800 = 60,700$\\
Initial tax: $(19,750)(0.10) + (60,700-19,750)(0.12) = 6889$\\
Apply credit: $6889 - 750 = 6139$
\end{center}

\item \text{}\\
\begin{tabular}{r l}
Taxpayer: & Married, filing jointly\\
Gross income: & \$52,000\\
Deductions: & \$9000: mortgage interest\\
& \$4500: charitable donations\\
& \$1500: theft loss\\
& \$1800: state taxes\\
Tax credit: & \$1400
\end{tabular} \answer{\$1469}
\begin{center}
Deductions: $9000 + 4500 + 1500 + 1800 = 16,800$ (lower than standard; use standard)\\
Taxable income: $52,000 - 24,800 = 27,200$\\
Initial tax: $(19,750)(0.10) + (27,200-19,750)(0.12) = 2869$\\
Apply credit: $2869 - 1400 = 1469$
\end{center}

\item \text{}\\
\begin{tabular}{r l}
Taxpayer: & Head of Household\\
Gross income: & \$104,000\\
Deductions: & \$18,000: mortgage interest\\
& \$5300: property taxes\\
& \$4800: state taxes\\
Tax credit: & none
\end{tabular} \answer{\$14,032.50}
\begin{center}
Deductions: $18,000 + 5300 + 4800 = 28,100$ (greater than standard)\\
Taxable income: $104,000 - 28,100 = 75,900$\\
Initial tax: $(14,100)(0.10) + (53,700-14,100)(0.12) + (75,900-40,125)(0.22) = 14,032.50$\\
Apply credit: N/A
\end{center}

\item \text{}\\
\begin{tabular}{r l}
Taxpayer: & Head of Household\\
Gross income: & \$43,000\\
Deductions: & \$3700: property taxes\\
& \$3650: state taxes\\
Tax credit: & none
\end{tabular} \answer{\$2640}
\begin{center}
Deductions: $3700 + 3650 = 7350$ (lower than standard; use standard)\\
Taxable income: $43,000 - 18,650 = 24,350$\\
Initial tax: $(14,100)(0.10) + (24,350-14,100)(0.12) = 2640$\\
Apply credit: N/A
\end{center}
\end{enumerate}