\begin{exercises}

\pthree{A \line(1,0){75} \ is a tax at a consistent rate.}
\pthree{A \line(1,0){75} \ is a tax for which the rate increases for higher taxed amounts}
\pthree{A \line(1,0){75} \ is a tax for which the rate decreases for higher taxed amounts}

\pthree{If the property taxes are \$1800 on a home valued at \$140,000, what is the effective tax rate?}
\pthree{In state A, the gas tax is 28 cents per gallon, where the average pre-tax cost of gas is \$2.58 per gallon.  In state B, the gas tax is 25 cents per gallon, where the average pre-tax cost of gas is \$2.50.  Which state has a lower gas tax rate?}
\pthree{If the sales tax is \$16.05 on a purchase of \$214, what is the sales tax rate?}

\ptwo{Using the 2020 tax table on page \pageref{Tax Table}, how much would a single taxpayer owe on a taxable income of \$55,000?}
\ptwo{Using the 2020 tax table on page \pageref{Tax Table}, how much would a married couple filing jointly owe on a taxable income of \$92,000?}

\textit{For problems 9--14, use the marginal tax table on page \pageref{Tax Table} to calculate the tax owed by each taxpayer.}\\
\ptwo{\\
\begin{tabular}{r l}
Taxpayer: & Single\\
Gross income: & \$75,000\\
Deductions: & \$18,000: mortgage interest\\
& \$2500: property taxes\\
& \$2000: charitable donations\\
& \$300: cost of tax preparation\\
Tax credit: & \$800
\end{tabular}}
\ptwo{\\
\begin{tabular}{r l}
Taxpayer: & Single\\
Gross income: & \$40,000\\
Deductions: & \$10,000: mortgage interest\\
& \$2000: property taxes\\
& \$300: charitable donations\\
Tax credit: & \$1300
\end{tabular}}

\ptwo{\\
\begin{tabular}{r l}
Taxpayer: & Married, filing jointly\\
Gross income: & \$85,500\\
Deductions: & \$5000: charitable donations\\
& \$3750: state taxes\\
Tax credit: & \$750
\end{tabular}}
\ptwo{\\
\begin{tabular}{r l}
Taxpayer: & Married, filing jointly\\
Gross income: & \$52,000\\
Deductions: & \$9000: mortgage interest\\
& \$4500: charitable donations\\
& \$1500: theft loss\\
& \$1800: state taxes\\
Tax credit: & \$1400
\end{tabular}}

\ptwo{\\
\begin{tabular}{r l}
Taxpayer: & Head of Household\\
Gross income: & \$104,000\\
Deductions: & \$18,000: mortgage interest\\
& \$5300: property taxes\\
& \$4800: state taxes\\
Tax credit: & none
\end{tabular}}
\ptwo{\\
\begin{tabular}{r l}
Taxpayer: & Head of Household\\
Gross income: & \$43,000\\
Deductions: & \$3700: property taxes\\
& \$3650: state taxes\\
Tax credit: & none
\end{tabular}}
\vfill
\pagebreak

\pone{\textbf{Project: Flat Tax, Modified Flat Tax, and Progressive Tax}

Many people have proposed various revisions to the income tax collection in the U.S.  Some, for example (including Milton Friedman), have claimed that a flat tax would be fairer.  Others call for revisions to how different types of income are taxed.  This project is for you to investigate this idea.\\

Imagine the country is made up of 100 households.  The federal government needs to collect \$800,000 in income taxes to be able to function (this is roughly proportional to the actual U.S. population and tax needs, but using smaller, more manageable numbers).  The population consists of 6 groups:
\begin{center}
\begin{tabular}{r l}
Group A: & 20 households that earn \$12,000 each\\
Group B: & 20 households that earn \$29,000 each\\
Group C: & 20 households that earn \$50,000 each\\
Group D: & 20 households that earn \$79,000 each\\
Group E: & 15 households that earn \$129,000 each\\
Group F: & 5 households that earn \$295,000 each
\end{tabular}
\end{center}
We are going to determine new income tax rates.

\paragraph{Proposal A} The first proposal we'll consider is a flat tax --- one where every income group is taxed at the same percentage rate.
\begin{enumerate}[1)]
\item Determine the total income for the population.
\vspace{1in}

\item Determine what flat tax rate would be necessary to collect enough money.
\vspace{1in}
\end{enumerate}

\paragraph{Proposal B} The second proposal we'll consider is a modified flat-tax plan, where everyone only pays taxes on any income over \$20,000.  So everyone is Group A will pay no taxes, for instance, and everyone in Group B will pay taxes only on \$9,000.
\begin{enumerate}[1)]
\setcounter{enumi}{2}
\item Determine the total \textit{taxable} income for the population.
\vspace{1in}

\item Determine what flat tax rate would be necessary to collect enough money in this modified system.
\end{enumerate}
}
\vfill
\pagebreak

\begin{minipage}[t]{\textwidth}
\begin{enumerate}[1)]
\setcounter{enumi}{4}
\item Complete the table below for both plans.
\end{enumerate}

\begin{center}
\begin{tabular}{|c | p{0.75in} | p{1.1in} | p{0.85in} | p{1.1in} | p{0.85in} |}
\hline
\multicolumn{2}{|c|}{} & \multicolumn{2}{c|}{Flat Tax Plan} & \multicolumn{2}{c|}{Modified Flat Tax Plan}\\
\hline
Group & Income per Household & Income Tax per Household & Income After Taxes & Income Tax per Household & Income After Taxes\\
\hline
& & & & & \\
A & \$12,000 & & & & \\
& & & & & \\
\hline
& & & & & \\
B & \$29,000 & & & & \\
& & & & & \\
\hline
& & & & & \\
C & \$50,000 & & & & \\
& & & & & \\
\hline
& & & & & \\
D & \$79,000 & & & & \\
& & & & & \\
\hline
& & & & & \\
E & \$129,000 & & & & \\
& & & & & \\
\hline
& & & & & \\
F & \$295,000 & & & & \\
& & & & & \\
\hline
\end{tabular}
\end{center}

\paragraph{Proposal C} The third proposal we'll consider is a progressive tax, where lower income groups are taxed at a lower percentage rate and higher income groups are taxed at a higher percentage rate.  For simplicity, we're going to assume that a household is taxed at the same rate on \textit{all} their income.
\begin{enumerate}[1)]
\setcounter{enumi}{5}
\item Set progressive tax rates for each income group to bring in enough money.  There is no single right answer here --- just make sure you bring in enough money (the total tax must add up to at least \$800,000)!\\
\begin{tabular}{|c | p{0.75in} | p{0.9in} | p{1.1in} | p{1.3in} | p{1.3in} |}
\hline
Group & Income per Household & Tax Rate (\%) & Income Tax per Household & Total Tax Collected for All Households & Income After Taxes per Household\\
\hline
& & & & & \\
A & \$12,000 & & & & \\
& & & & & \\
\hline
& & & & & \\
B & \$29,000 & & & & \\
& & & & & \\
\hline
& & & & & \\
C & \$50,000 & & & & \\
& & & & & \\
\hline
& & & & & \\
D & \$79,000 & & & & \\
& & & & & \\
\hline
& & & & & \\
E & \$129,000 & & & & \\
& & & & & \\
\hline
& & & & & \\
F & \$295,000 & & & & \\
& & & & & \\
\hline
\end{tabular}
\end{enumerate}
\end{minipage}

\begin{minipage}[t]{\textwidth}
\begin{enumerate}[1)]
\setcounter{enumi}{6}
\item Discretionary income is the income people have left over after paying for necessities like rent, food, transportation, etc.  The cost of basic expenses does increase with income, since housing and car costs are higher.  However, these increases are usually not proportional to the increase in income.  For each income group, estimate their essential expenses and calculate their discretionary income.  Then compute the effective tax rate for each plan relative to discretionary income rather than income.
\begin{center}
\begin{tabular}{|c | p{0.75in} | p{1.3in} | p{0.75in} | p{1in} | p{1in} |}
\hline
Group & Income per Household & Discretionary Income (estimated) & Effective Rate, Flat & Effective Rate, Modified & Effective Rate, Progressive\\
\hline
& & & & & \\
A & \$12,000 & & & & \\
& & & & & \\
\hline
& & & & & \\
B & \$29,000 & & & & \\
& & & & & \\
\hline
& & & & & \\
C & \$50,000 & & & & \\
& & & & & \\
\hline
& & & & & \\
D & \$79,000 & & & & \\
& & & & & \\
\hline
& & & & & \\
E & \$129,000 & & & & \\
& & & & & \\
\hline
& & & & & \\
F & \$295,000 & & & & \\
& & & & & \\
\hline
\end{tabular}
\end{center}

\item Which plan seems the most fair to you?  Which plan seems the least fair to you?  Why?
\end{enumerate}
\end{minipage}
\end{exercises}