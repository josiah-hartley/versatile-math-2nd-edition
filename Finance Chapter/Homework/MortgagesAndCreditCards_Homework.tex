\begin{exercises}
\ptwo{If you take out an auto loan of \$8500 at 5\% interest for 48 months, what will your monthly payment be?}
\ptwo{If you borrow \$13,000 to buy a boat, and the bank charges 7\% interest for 72 months, how much will you have to pay each month?}

\ptwo{Janine bought \$3000 of new furniture on credit.  Because her credit score isn't very good, the store is charging her a fairly high interest rate on the loan: 16\%.  If she agreed to pay off the furniture over two years, how much will she have to pay each month?}
\ptwo{Carly financed a new \$1200 television at 12\% for 48 months.  How much will she have to pay every month to pay this off?}

\ptwo{If you want to buy a car, and you can afford a monthly payment of \$175, how large of a loan can you get at 4.8\% interest over 60 months?}
\ptwo{Mary is going to finance new office equipment at a 2\% rate over a 4 year term.  If she can afford monthly payments of \$100, how much can she pay for the new office equipment?}

\ptwo{If you buy a \$33,000 car for \$1000 down and monthly payments of \$685 for 60 months, how much will you pay in total for the car?}
\ptwo{A car costs \$27,000, and you're offered a loan that requires \$800 down and a monthly payment of \$575 for 60 months, how much will you pay in interest?}

\ptwo{A car dealership offers a loan with 3.5\% interest for 60 months, and you plan to purchase a car for \$18,000.  You can afford a down payment of \$2500.
\begin{enumerate}[(a)]
\item What will your monthly payment be?
\item How much will you pay in total for the car?
\item How much will you pay in interest over the life of the loan?
\end{enumerate}}
\ptwo{You plan to purchase a \$21,000 car, and your bank offers you a loan at 4.5\% interest for 48 months.  You can afford a down payment of \$4000.
\begin{enumerate}[(a)]
\item What will your monthly payment be?
\item How much will you pay in total for the car?
\item How much will you pay in interest over the life of the loan?
\end{enumerate}}

\ptwo{You want to buy a \$200,000 home, and you have \$40,000 saved up.  The bank offers a 30-year mortgage at 3.8\% interest.
\begin{enumerate}[(a)]
\item If you expect to pay \$6000 in closing costs, what percentage down payment can you afford?
\item If you put less than 20\% down, you'll need to pay mortgage insurance.  Will you require mortgage insurance?
\item What will the principal be on the loan?
\item What will your monthly P\&I payment be?
\item In addition to principal and interest, the property taxes will be 1.5\% of the home value per year, homeowners insurance will be \$750 per year, and the mortgage insurance (if needed, according to part (b)) will be \$25 per month.  What will your total monthly payment amount be?
\item How much will you pay in total over 30 years in principal and interest?
\item How much interest will you pay in total?
\end{enumerate}}
\ptwo{You want to buy a \$375,000 home, and you have \$84,000 saved up.  The bank offers a 20-year mortgage at 3.2\% interest.
\begin{enumerate}[(a)]
\item If you expect to pay \$8000 in closing costs, what percentage down payment can you afford?
\item If you put less than 20\% down, you'll need to pay mortgage insurance.  Will you require mortgage insurance?
\item What will the principal be on the loan?
\item What will your monthly P\&I payment be?
\item In addition to principal and interest, the property taxes will be 1.5\% of the home value per year, homeowners insurance will be \$825 per year, and the mortgage insurance (if needed, according to part (b)) will be \$30 per month.  What will your total monthly payment amount be?
\item How much will you pay in total over 20 years in principal and interest?
\item How much interest will you pay in total?
\end{enumerate}}

\ptwo{You can afford a \$900 per month mortgage payment.  You've found a 30-year loan at 4\% interest.
\begin{enumerate}[(a)]
\item How big of a loan can you afford?
\item How much total money will you pay the bank?
\item How much of that money is interest?
\end{enumerate}}
\ptwo{You can afford a \$1790 per month mortgage payment.  You've found a 15-year loan at 3.25\% interest.
\begin{enumerate}[(a)]
\item How big of a loan can you afford?
\item How much total money will you pay the bank?
\item How much of that money is interest?
\end{enumerate}}

\pthree{If the interest rate on a 30-year mortgage for \$175,000 were changed from 3.8\% to 3.1\%, how much would you save over the life of the loan?}
\pthree{How much would you save (over the life of the loan) on a 20-year mortgage at 4.5\% if you reduced the amount you borrowed from \$300,000 to \$260,000?}
\pthree{If you borrow \$250,000, how much could you save over the life of the loan if you took out a 15-year mortgage at 4.5\% instead of a 30-year mortgage at 4\%?}

\pone{Suppose you take out a \$315,000 mortgage for 30 years at 4.5\% interest.
\begin{enumerate}[(a)]
\item Find the monthly payment on this mortgage.
\item Fill out the first two rows of the amortization schedule below.
\begin{center}
\begin{tabular}{|>{\centering\arraybackslash\hspace{0pt}}p{1.1in} | >{\centering\arraybackslash\hspace{0pt}}p{1.1in} | >{\centering\arraybackslash\hspace{0pt}}p{1.2in} | >{\centering\arraybackslash\hspace{0pt}}p{1in}|}
\hline
{\small Payment Number} & {\small Interest Payment} & {\small Principal Payment} & {\small Balance of Loan}\\
\hline
1 & & & \\
\hline
2 & & & \\
\hline
& & &
\end{tabular}
\end{center}
\end{enumerate}}

\pone{Suppose you take out a \$180,000 mortgage for 15 years at 3.7\% interest.
\begin{enumerate}[(a)]
\item Find the monthly payment on this mortgage.
\item Fill out the first two rows of the amortization schedule below.
\begin{center}
\begin{tabular}{|>{\centering\arraybackslash\hspace{0pt}}p{1.1in} | >{\centering\arraybackslash\hspace{0pt}}p{1.1in} | >{\centering\arraybackslash\hspace{0pt}}p{1.2in} | >{\centering\arraybackslash\hspace{0pt}}p{1in}|}
\hline
{\small Payment Number} & {\small Interest Payment} & {\small Principal Payment} & {\small Balance of Loan}\\
\hline
1 & & & \\
\hline
2 & & & \\
\hline
& & &
\end{tabular}
\end{center}
\end{enumerate}}

\ptwo{Suppose your VISA card calculates interest using the average daily balance method, and the monthly interest rate is 1.4\%.  The itemized billing for the month of April is shown below.\\

\begin{tabular}{l l l}
Detail & Date & Amount\\
\hline
Unpaid balance & April 1 & \$1100\\
Payment received & April 3 & \$500\\
New computer & April 11 & \$750\\
Books & April 15 & \$65\\
Mattress & April 28 & \$600\\
Last day of billing period & April 30 &\\
Payment Due Date & May 7 &\\
\end{tabular}

\begin{enumerate}[(a)]
\item Find the average daily balance.
\item Find the interest due for this month.
\item Find the total balance owed on the last day of the billing period.
\item This credit card requires a \$20 minimum payment or 1/36 of the amount due, whichever is higher.  What is the minimum monthly payment due for this month?
\end{enumerate}}
\ptwo{Suppose your MasterCard calculates interest using the average daily balance method, and the monthly interest rate is 2.1\%.  The itemized billing for the month of August is shown below.\\

\begin{tabular}{l l l}
Detail & Date & Amount\\
\hline
Unpaid balance & August 1 & \$300\\
Payment received & August 9 & \$100\\
Tuition & August 10 & \$4500\\
Textbooks & August 18 & \$350\\
Groceries & August 25 & \$180\\
Last day of billing period & August 31 &\\
Payment Due Date & September 7 &\\
\end{tabular}

\begin{enumerate}[(a)]
\item Find the average daily balance.
\item Find the interest due for this month.
\item Find the total balance owed on the last day of the billing period.
\item This credit card requires a \$15 minimum payment or 1/24 of the amount due, whichever is higher.  What is the minimum monthly payment due for this month?
\end{enumerate}}
\vfill
\pagebreak

\pone{\textbf{Project: Finding a Mortgage}

You and your family are looking to move and are shopping for a house.  Your job is to find a mortgage that you can afford.

You may choose your family size---you can be married with kids, married without kids, or single.  You may also pick anywhere in the country that you'd like to live, but you can only make the median income listed for the state you choose.  If you are married, you can assume that both you and your spouse are working and each are paid the median income for that state.

\begin{enumerate}[(a)]
\item Decide where you want to live.  Do some research and find the median income for that state, and decide whether you are single or married, and whether or not you have children.
\item Search \href{http://www.realtor.com}{realtor.com} or a similar website to find a house that fits your family's needs.  Take note of the
\begin{itemize}
\item List price of the home.
\item Property taxes listed under the ``Property History'' tab.  If property taxes are not listed, estimate the annual property taxes as 2\% of the purchase price.
\end{itemize}
\item Estimate the down payment you can afford, and take note of the principal of the loan that you will need.
\item Do some research to find current interest rates.  Use \href{http://www.bankrate.com}{bankrate.com} or a similar website to find mortgage rates (make sure to find a mortgage without \emph{points}).  Find three options: mortgages for 30 years, 20 years, and 15 years.
\begin{itemize}
\item What is the monthly payment for each option?
\item How much will you pay in total for principal and interest over the life of the loan for each option?
\item Keeping your monthly budget in mind, which option will you choose?
\end{itemize}

\item Complete the following steps to find if you can afford this home.  This worksheet uses a typical rule of thumb that you should not spend more than 28\% of your income on housing.
\begin{center}
\begin{tabular}{p{3.75in} p{3in}}
\textbf{Monthly Gross Income} &\\
\hspace{0.5in} Borrower's annual income & $\$ \line(1,0){150}$\\
& \\
\hspace{0.5in} Co-borrower's annual income & $+ \line(1,0){150}$\\
& \\
\hspace{0.5in} Total gross annual income & $\$ \line(1,0){150}$\\
\hspace{0.75in} Divide total gross income by 12 & \hspace{0.25in} $\div$ 12\\
& \\
\hspace{0.5in} Total monthly gross income & $\$ \line (1,0){150}$\\
\hspace{0.75in} Find 28\% of this & \hspace{0.25in} $\times 0.28$\\
& \\
\hspace{0.5in} \textbf{Allowable monthly housing cost} & $\$\line(1,0){150}$ (A)\\
& \\
\textbf{Monthly Taxes} &\\
\hspace{0.5in} Home purchase price & $\$ \line(1,0){150}$\\
& \\
\hspace{0.5in} Estimated taxes & $\$ \line(1,0){150}$\\
\hspace{0.75in} Divide taxes by 12 & \hspace{0.25in} $\div 12$\\
& \\
\hspace{0.5in} Monthly taxes & $\$ \line(1,0){150}$ (B)\\
& \\
\textbf{Monthly Housing Cost} & \\
\hspace{0.5in} Monthly mortgage payment & $\$ \line(1,0){150} \ +$\\
& \\
\hspace{0.5in} Estimated monthly taxes (B) & $\$ \line(1,0){150} \ +$\\
& \\
\hspace{0.5in} Condo or homeowner's fee (if applicable) & $\$ \line(1,0){150}$\\
& \\
\textbf{Total Monthly Housing Cost} & $= \$ \line(1,0){140}$ (C)
\end{tabular}
\end{center}
Compare (A) and (C).  Can you afford the house you want to buy?  If not, choose a less expensive house and redo this project.
\end{enumerate}}

\end{exercises}