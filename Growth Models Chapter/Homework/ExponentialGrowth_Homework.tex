\begin{exercises}
\emph{In problems 1--4, use a calculator to solve for the unknown variable, $x$ or $t$.}\\
\pfour{$10^x=5$}
\pfour{$7^t=100$}
\pfour{$4(10^x)=9$}
\pfour{$22(1.065)^{0.05t}=37$}

\ptwo{The population of the District of Columbia was approximately 572,000 in 2000, and has been growing at a rate of about 1.15\%.
\begin{enumerate}[(a)]
\item Write an exponential model of the form $P_t = P_0(1 + r)^t$ to describe the population of DC from 2000 onward.
\item If this trend continues, what will DC's population be in 2025?
\item When does this model predict that DC's population will reach 800,000?
\end{enumerate}}
\ptwo{Baltimore's population in 2010 was approximately 620,000, and has been decreasing at a rate of about 0.5\% per year.
\begin{enumerate}[(a)]
\item Write an exponential model of the form $P_t = P_0(1 + r)^t$ to describe the population of Baltimore from 2010 onward.
\item If this trend continues, what will Baltimore's population be in 2030?
\item When does this model predict that Baltimore's population will reach 500,000?
\end{enumerate}}

\ptwo{Diseases tend to spread exponentially.  In the early days of AIDS, the growth rate was around 190\%.  In 1983, about 1700 people in the US died of AIDS.  If the trend had continued unchecked, how many people would have died from AIDS in 1990?}
\ptwo{The population of the world in 1987 was 5 billion and the annual growth rate was estimated at 2 percent.  If the world population followed an exponential growth model, find the projected world population in 2015.}

\ptwo{The population of Maryland was 5.17 million in 1999, and it grew to 6.05 million in 2019.
\begin{enumerate}[(a)]
\item Assuming that the population is growing exponentially, find the growth rate $r$ for Maryland's population.
\item Write an exponential model to describe the population of Maryland from 1999 onward.
\item What is Maryland's population expected to be in 2030?
\item When do you expect that Maryland's population will reach 8 million?
\end{enumerate}}
\ptwo{The population of Virginia was 6.87 million in 1999, and it grew to 8.54 million in 2019.
\begin{enumerate}[(a)]
\item Assuming that the population is growing exponentially, find the growth rate $r$ for Virginia's population.
\item Write an exponential model to describe the population of Virginia from 1999 onward.
\item What is Virginia's population expected to be in 2022?
\item When do you expect that Virginia's population will reach 10 million?
\end{enumerate}}

\ptwo{A bacteria culture is started with 300 bacteria.  After 4 hours, the population has grown to 500 bacteria.  If the population grows exponentially according to the formula $P_t = P_0(1+r)^t$,
\begin{enumerate}[(a)]
\item Find the growth rate $r$ and write the full formula.
\item If this trend continues, how many bacteria will there be in one day?
\item How long will it take for this culture to triple in size?
\end{enumerate}}
\ptwo{A native wolf species has been reintroduced into a national forest.  Originally 200 wolves were transplanted, and after 3 years, the population had grown to 270 wolves.  If the population grows exponentially according to the formula $P_t = P_0(1+r)^t$,
\begin{enumerate}[(a)]
\item Find the growth rate $r$ and write the full formula.
\item If this trend continues, how many wolves will there be in ten years?
\item If this trend continues, how long will it take the wolf population to double?
\end{enumerate}}

\ptwo{In 2009, the average compensation for CEOs in the U.S. was approximately \$10,800,000, and by 2016, this had risen to about \$12,800,000.  By comparison, the average compensation for workers was \$54,700 in 2009 and \$55,800 in 2016.  Assume that both values are growing according to an exponential model.  Find the growth rate for both salaries; which is higher?}
\ptwo{In 2008, approximately 131 million people voted in the U.S. general election, compared to about 139 million people in 2016.  The total population of the U.S. was 304 million in 2008 and 323 million in 2016.  Assume that both levels are growing exponentially.  Find the growth rate for both populations; which is higher?}

\ptwo{The table below shows the population of Canada from 2010 to 2019.
\begin{center}
\begin{tabular}{c c}
\textbf{Year} & \textbf{Population (millions)}\\
\hline
 & \\
2010 & 34.0\\
2011 & 33.5\\
2012 & 34.7\\
2013 & 35.1\\
2014 & 35.4\\
2015 & 35.7\\
2016 & 35.1\\
2017 & 36.5\\
2018 & 37.1\\
2019 & 37.6
\end{tabular}
\end{center}
\begin{enumerate}[(a)]
\item Use a graphing calculator or spreadsheet program to build an exponential regression model, letting $t=0$ in 2010.
\item What does this model predict that the population of Canada will be in 2035?
\item When does this model predict that Canada's population will reach 40 million?
\end{enumerate}}
\ptwo{The table below shows the population of Mexico from 2010 to 2019.
\begin{center}
\begin{tabular}{c c}
\textbf{Year} & \textbf{Population (millions)}\\
\hline
 & \\
2010 & 114.1\\
2011 & 115.7\\
2012 & 117.3\\
2013 & 118.8\\
2014 & 120.4\\
2015 & 121.9\\
2016 & 123.3\\
2017 & 124.8\\
2018 & 126.2\\
2019 & 127.6
\end{tabular}
\end{center}
\begin{enumerate}[(a)]
\item Use a graphing calculator or spreadsheet program to build an exponential regression model, letting $t=0$ in 2010.
\item What does this model predict that the population of Mexico will be in 2040?
\item When does this model predict that Mexico's population will reach 145 million?
\end{enumerate}}

\end{exercises}