\begin{enumerate}
\item The table below shows the distance that a baseball travels after being hit at various angles.
\begin{center}
\begin{tabular}{c c}
\textbf{Angle (degrees)} & \textbf{Distance (feet)}\\
\hline
 & \\
10 & 115.6\\
15 & 157.2\\
20 & 189.2\\
24 & 220.8\\
30 & 253.8\\
34 & 269.2\\
40 & 284.8\\
45 & 285.0\\
48 & 277.4\\
50 & 269.2\\
58 & 244.2\\
60 & 231.4\\
64 & 180.4
\end{tabular}
\end{center}
\begin{enumerate}[(a)]
\item Use a graphing calculator or spreadsheet program to find a quadratic model that best fits this data, using angle as $x$ and distance as $y$. \answersub{$y=-0.174x^2 + 14.521x - 21.898$}
\item Based on this model, what distance is expected for a ball hit at 55$^\circ$? \answersub{250.4 feet}
\[y = -0.174(55)^2 + 14.521(55) - 21.898 = 250.41\]
\item What distance is expected for a ball hit at 75$^\circ$? \answersub{88.4 feet}
\[y = -0.174(75)^2 + 14.521(75) - 21.898 = 88.43\]
\item Which of the two previous predictions is likely to be more reliable? \answersub{The one for 55$^\circ$; interpolation instead of extrapolation}
\item What angle above 45$^\circ$ would you expect to yield a distance of 200 feet? \answersub{63.3$^\circ$}
\begin{center}
Used calculator intersect function
\end{center}
\end{enumerate}

\item A ball is dropped from a height of a little over 5 feet, and the height is measured at small intervals.  The table below shows the results.
\begin{center}
\begin{tabular}{c c}
\textbf{Time (seconds)} & \textbf{Height (feet)}\\
\hline
 & \\
0.00 & 5.235\\
0.04 & 5.160\\
0.08 & 5.027\\
0.12 & 4.851\\
0.16 & 4.631\\
0.20 & 4.357\\
0.24 & 4.030\\
0.28 & 3.655\\
0.32 & 3.234\\
0.36 & 2.769\\
0.40 & 2.258\\
0.44 & 1.635
\end{tabular}
\end{center}
\begin{enumerate}[(a)]
\item Use a graphing calculator or spreadsheet program to find a quadratic model that best fits this data, using time as $x$ and height as $y$. \answersub{$y = -15.64x^2 - 1.24x + 5.23$}
\item Based on this model, what height is expected after 0.30 seconds? \answersub{3.450 feet}
\[y = -15.64(0.3)^2 - 1.24(0.3) + 5.23 = 3.450\]
\item What height is expected after 0.52 seconds? \answersub{0.356 feet}
\[y = -15.64(0.52)^2 - 1.24(0.52) + 5.23 = 0.356\]
\item Which of the two previous predictions is likely to be more reliable? \answersub{The one for 0.3 seconds; interpolation instead of extrapolation}
\item When do you expect the height of the ball to be 1 foot? \answersub{0.48 seconds}
\begin{center}
Used calculator intersect function
\end{center}
\end{enumerate}

\item The table below shows the amount spent on movie theater tickets in the U.S. from 1997 to 2003.
\begin{center}
\begin{tabular}{c c}
\textbf{Year} & \textbf{Spending (billions of dollars)}\\
\hline
 & \\
1997 & 6.3\\
1998 & 6.9\\
1999 & 7.9\\
2000 & 8.6\\
2001 & 9.0\\
2002 & 9.6\\
2003 & 9.9
\end{tabular}
\end{center}
\begin{enumerate}[(a)]
\item Use a graphing calculator or spreadsheet program to find a quadratic model that best fits the data.  Let $t$ represent the year, with $t=0$ in 1997. \answersub{$P_t = -0.049t^2 + 0.911t + 6.217$}
\item Based on this model, how much would you expect to be spent on movie theater tickets in 2008? \answersub{\$10.3 billion}
\[P_t = -0.049(11)^2 + 0.911(11) + 6.217 = 10.3\]
\item When would you expect movie theater ticket expenditure to fall to \$5 billion? \answersub{$t = 19.8$; late 2016}
\begin{center}
Used calculator intersect function
\end{center}
\end{enumerate}

\item The table below shows the number of FM radio stations in the U.S. from 1997 to 2003.
\begin{center}
\begin{tabular}{c c}
\textbf{Year} & \textbf{Stations}\\
\hline
 & \\
1997 & 5542\\
1998 & 5662\\
1999 & 5766\\
2000 & 5892\\
2001 & 6051\\
2002 & 6161\\
2003 & 6207
\end{tabular}
\end{center}
\begin{enumerate}[(a)]
\item Use a graphing calculator or spreadsheet program to find a quadratic model that best fits the data.  Let $t$ represent the year, with $t=0$ in 1997. \answersub{$P_t = -3.26t^2 + 136.64t + 5529.76$}
\item Based on this model, how many FM stations would you expect there to be in 2010? \answersub{6755 stations}
\[P_t = -3.26(13)^2 + 136.64(13) + 5529.76 = 6755.1\]
\item When would you expect the number of stations to first reach 6500? \answersub{$t = 9.1$; early 2006}
\begin{center}
Used calculator intersect function
\end{center}
\end{enumerate}

\item The table below shows college textbook sales in the U.S. from 2000 to 2005.
\begin{center}
\begin{tabular}{c c}
\textbf{Year} & \textbf{Textbook Sales (millions of dollars)}\\
\hline
 & \\
2000 & 4265\\
2001 & 4571\\
2002 & 4899\\
2003 & 5086\\
2004 & 5479\\
2005 & 5703
\end{tabular}
\end{center}
\begin{enumerate}[(a)]
\item Use a graphing calculator or spreadsheet program to find a quadratic model that best fits the data.  Let $t$ represent the year, with $t=0$ in 2000. \answersub{$P_t = -2.68t^2 + 301.99t + 4270.07$}
\item Based on this model, how much would you expect to be spent on college textbooks in 2015? \answersub{\$8197 million, or just over \$8 billion}
\[P_t = -2.68(15)^2 + 301.99(15) + 4270.07 = 8196.9\]
\item When would you expect textbook sales to first reach \$7 billion (\$7000 million)? \answersub{$t = 9.9$; late 2009}
\begin{center}
Used calculator intersect function
\end{center}
\end{enumerate}

\item The table below shows the average amount of time spent per person on entertainment per year from 2000 to 2005.
\begin{center}
\begin{tabular}{c c}
\textbf{Year} & \textbf{Hours}\\
\hline
 & \\
2000 & 3492\\
2001 & 3540\\
2002 & 3606\\
2003 & 3663\\
2004 & 3757\\
2005 & 3809
\end{tabular}
\end{center}
\begin{enumerate}[(a)]
\item Use a graphing calculator or spreadsheet program to find a quadratic model that best fits the data.  Let $t$ represent the year, with $t=0$ in 2000. \answersub{$P_t = 2.36t^2 + 53.73t + 3488.57$}
\item Based on this model, how many hours would you expect the average person to spend on entertainment in 2012? \answersub{4473 hours}
\[P_t = 2.36(12)^2 + 53.73(12) + 3488.57 = 4473.2\]
\item When would you expect the average amount of time to reach 4000? \answersub{$t=7.2$; early 2007}
\begin{center}
Used calculator intersect function
\end{center}
\end{enumerate}
\end{enumerate}