\begin{enumerate}
\item One hundred trout are seeded into a lake.  Absent constraint, their population will grow by 70\% a year.  If the lake can sustain a maximum of 2000 trout, use a logistic growth model to estimate the number of trout after 2 years. \answer{352 trout}
\[P_t = \dfrac{2000}{1 + \left(\dfrac{2000}{100} - 1\right)e^{-(0.7)(2)}} = 352\]

\item Ten blackberry plants started growing in a yard.  Absent constraint, blackberries will spread by 200\% a month.  If the yard can only sustain 50 plants, use a logistic growth model to estimate the number of plants after 2 months. \answer{47 plants}
\[P_t = \dfrac{50}{1 + \left(\dfrac{50}{10} - 1\right)e^{-(2)(2)}} = 47\]

\item A certain community consists of 1000 people, and one individual has a particularly contagious strain of influenza.  Assuming the community has not had vaccination shots and are all susceptible, the spread of the disease in the community is modeled by \[P_t = \dfrac{1000}{1+999e^{-0.3t}}\] where $P_t$ is the number of people who have contracted the flu after $t$ days.
\begin{enumerate}[(a)]
\item How many people have contracted the flu after 10 days?  Round your answer to the nearest whole number. \answersub{20 people}
\[P_t = \dfrac{1000}{1+999e^{-0.3(10)}} = 20\]
\item What is the carrying capacity for this model?  Does this make sense? \answersub{1000 people (the size of the community)}
\item How many days will it take for 750 people to contract the flu?  Round your answer to the nearest whole number. \answersub{27 days}
\[750 = \dfrac{1000}{1+999e^{-0.3t}} \longrightarrow t = 26.68 \textrm{ (used calculator)}\]
\end{enumerate}

\item A herd of 20 white-tailed deer is introduced to a coastal island where there had been no deer before.  Their population is predicted to increase according to \[P_t = \dfrac{100}{1+4e^{-0.14t}}\] where $P_t$ is the number of deer expected in the herd after $t$ years.
\begin{enumerate}[(a)]
\item How many deer will be present after 2 years?  Round your answer to the nearest whole number. \answersub{25 deer}
\[P_t = \dfrac{100}{1+4e^{-0.14(2)}} = 24.9\]
\item What is the carrying capacity for this model? \answersub{100 deer}
\item How many years will it take for the herd to grow to 50 deer?  Round your answer to the nearest whole number. \answersub{10 years}
\[50 = \dfrac{100}{1+4e^{-0.14t}} \longrightarrow t = 9.9 \textrm{ (used calculator)}\]
\end{enumerate}

\item The table below shows the population of California from 2010 to 2019.
\begin{center}
\begin{tabular}{c c}
\textbf{Year} & \textbf{Population (millions)}\\
\hline
 & \\
2010 & 37.3\\
2011 & 37.6\\
2012 & 38.0\\
2013 & 38.3\\
2014 & 38.6\\
2015 & 38.9\\
2016 & 39.2\\
2017 & 39.4\\
2018 & 39.5\\
2019 & 39.5
\end{tabular}
\end{center}
\begin{enumerate}[(a)]
\item Use a graphing calculator to build a logistic regression model, letting $t=0$ in 2010. \answersub{$P_t = \dfrac{40.591}{1+0.090e^{-0.145t}}$}
\item What does this model predict that the population of California will be in 2025? \answersub{40.2 million}
\[P_t = \dfrac{40.591}{1+0.090e^{-0.145(15)}} = 40.2\]
\item When does this model predict that California's population will reach 40 million? \answersub{$t = 12.5$; midway through 2022}
\[40 = \dfrac{40.591}{1+0.090e^{-0.145t}} \longrightarrow t = 12.5 \textrm{ (used calculator)}\]
\item According to this model, what is the carrying capacity for California's population? \answersub{about 40.6 million}
\end{enumerate}

\item The table below shows the population of Florida from 2010 to 2019.
\begin{center}
\begin{tabular}{c c}
\textbf{Year} & \textbf{Population (millions)}\\
\hline
 & \\
2010 & 18.7\\
2011 & 19.1\\
2012 & 19.3\\
2013 & 19.6\\
2014 & 19.9\\
2015 & 20.2\\
2016 & 20.6\\
2017 & 21.0\\
2018 & 21.2\\
2019 & 21.5
\end{tabular}
\end{center}
\begin{enumerate}[(a)]
\item Use a graphing calculator to build a logistic regression model, letting $t=0$ in 2010. \answersub{$P_t = \dfrac{93.286}{1+3.983e^{-0.020t}}$}
\item What does this model predict that the population of Florida will be in 2030? \answersub{25.4 million}
\[P_t = \dfrac{93.286}{1+3.983e^{-0.020(20)}} = 25.4\]
\item When does this model predict that Florida's population will reach 23 million? \answersub{$t = 13.2$; early 2023}
\[23 = \dfrac{93.286}{1+3.983e^{-0.020t}} \longrightarrow t = 13.2 \textrm{ (used calculator)}\]
\item According to this model, what is the carrying capacity for Florida's population? \answersub{about 93.3 million}
\end{enumerate}
\end{enumerate}